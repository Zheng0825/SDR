Software Defined Radio Networks (SDRNs)  use systems of Software Defined Radios (SDRs) to establish
networks with flexible physical and link layers. GNU Radio is an open source software
tool set for working with SDRs and can be used as a basis for creating SDRNs.
Mesh networks are designed to allow for flexible and 
distributed network architectures that are self forming and function without
the need for centralized infrastructure. Batman-adv is a popular, open source, layer 2 mesh network
protocol. Our work establishes an SDRN platform by combining GNU Radio with Batman-adv to create a fully open
source software defined radio mesh network. The platform can work with USRP SDR devices to quickly
prototype and experiment with SDR and cognitive radio frameworks. Due to the flexibility of Batman-adv
and GNU Radio, programs acting above Layer 2 can utilize this network without any changes. We further
increase the cognitive abilities of the platform by leveraging the A.L.F.R.E.D. tool within Batman-adv
to distribute information about frequency changes across the mesh network.  

