Cognitive Radio Networks (CRNs) seek to use systems of Software Defined Radios (SDRs) to establish
networks with flexible and intelligent physical and link layers. GNU Radio is an open source software
tool set for working with SDRs and can be used as a basis for creating CRNs.
Mesh networks are designed to allow for flexible and 
distributed network architectures to be self forming and to function without
the need for centralized infrastructure. Batman-adv is a popular open source layer 2 mesh network
protocol. Our work established a CRN by combinging GNU Radio with Batman-adv to create a fully open
source cognitivie radio mesh network. The platform can work with USRP SDR devices to quickly
prototype and experiment with cogntivie radio frameworks. Due to the flexibilty of Batman-adv
and GNU Radio, programs acting above Layer 2 can utilize this network without any changes. We further
increase the cognitive abilities of the platform by leveraging the A.L.F.R.E.D. tool within Batman-adv
to distribute information about frequency changes across the mesh network.  

