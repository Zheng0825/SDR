\section{Introduction}

Software defined radios (SDRs) have been around for many years. However, as the cost of SDRs
continues to drop, the technology becomes much more accessible. Additionally, open source
tools such as GNU Radio make developing for SDRs much easier. GNU Radio is an open source
project to create an easy to use tool chain for creating SDR Projects. The GNU Radio
Companion allows for GUI development of PHY and MAC layer protocols. The project itself
is implemented in a combination of Python and C++ modules. Ettus research, a division of 
National Instruments, created the Universal Software Radio Peripheral (USRP) and the Universal
Hardware Drivers (UHD). These tools have been integrated into the GNU Radio ecosystem, by
Ettus and other developers.  

Cognitive Radio Networks (CRNs) are networks made up of SDRs that are capable of making 
intelligent decisions on their own and adjusting parameters such as signal strength and 
operating frequency. Many Cognitive Radio scenarios are designed around the idea of ad-hoc or mesh
networks. In these networks, all of the associated radio components are able to talk to each
other either directly or by "hopping" from one node to another until they reach their destination. 
The Better Approach to Mobile Ad-hoc Networks (BATMAN) project created the Batman-adv protocol. 
This layer 2 protocol has a fairly large community and is integrated into the Linux Kernel and OpenWRT
project. 

