\documentclass{IEEEtran}

\usepackage{hyperref}
\usepackage{listings}
\usepackage{graphicx}
\graphicspath{{img/}}

\title{Abstract and Overview of Topics for Graduate Research}
\date{2015-09-11}
\author{John McCormack}

\begin{document}
\maketitle

\begin{abstract}
	This document serves to organizes thoughts and ideas about
the remainder of my time at Florida Polytechnic. It discusses areas of interest
that I would like to explore, and hopes to find projects that will give me
exposure to the topics. 
\end{abstract}
\begin{IEEEkeywords}
Software Defined Radio, Mesh Network, HCI/HMI, Machine Learning, NodeJs, 
Internet of Things, Robotics, Serious Games
\end{IEEEkeywords}
\section{SDR}
Currently, GNU Radio has been used successfully with the RTL-SDR. The biggest
issue is that this SDR is extremely minimalistic and cannot be used for our
long term goals. It has a limited useable band, and can only recieve. The next
step is to continue working with the BladeRF and GNURadio. Right now, there is an 
issue with the DC offset of the chip which is causing difficulties. 

Long term, we need the SDR to be able to act as the Wi-Fi card of the computer. 
We also need to then develop a model for sensing the ambient RF activity and
autonomously adjusting the transmit and recieve frequency. 

Additionally, testing needs to be done to determine if the Two Yagi antennas are
able to be used in Stereo. 

\section{Mesh Networks}

The mesh network is currently in an early alpha stage. It is functioning, but
is lacking of any proper documentation and it would not be easy to rebuild
with out help from students that are no longer with the school. 

Most of my personal interests around the mesh network would involve embedding it 
into smaller packages so that it could be manufactured as part of an IoT SoC/SoP. 
However, that would likely fall much too far out of the scope of what I would be
able to accomplish while at florida polytechnic. 

Most of the interest with Mesh Networks lies in using it with an SDR, visualization
tool, or IoT Application. I am interested in the basic configuration, but would
like to focus on the applications more than the implementation. 

\section{Internet of Things}

I am very interested in applications of Internet of Things. I would like to
pursue Contiki, TinyOs, or RiotOs. I believe that these tools, along with Angular,
NodeJs, Flask, and other similar tools, I could put myself in a good position
with potential employers should I ultimately decide to not pursue a PhD program. 

I am very interested in the integration of these tools with Unity for visualization
and control. 

\section{Robotics}

As robotics is such a broad topic, it is hard to find a small enough chunk
that could be doable alongside the other topics presented. In the context
of the rest of the other research areas, communication makes the most sense. 
Robots could be used alongside the mesh network to allow for remote control. 
SDRs could also be deployed on top of some of the robots we have to allow
for a dynamic mesh that is capable of changing shapes with the demands of the
network. 

As a side project, I would like to implement SLAM on a GoPiGo and possibly store
the map on the mesh for other robots and the sensor nodes themselves to reference.
It would be interesting to place the nodes, then have a robot map out the space,
marking where each node was in order to then use both the map and the presence of
nodes to allow other robots to perform localization. 

\section{Machine Learning}
Machine learning would be performed alongside the SDRs. This will probably be
very basic to not take away from the other topics. If time permitted,
using Erlang with the mesh nodes to create a distributed ANN would be intriguing. 

\section{Serious Games}

I would like to continue working with Unity to map out the mesh nodes. I 
would also like to expand to using it to control some of the less complex
robots in the lab. Eventually, if time permits and there is student interest,
I would like to return to the King of the Mesh concept. 

\section{Various Programming Languages and Tools}

\begin{itemize}
	\item NodeJs - Continue Progress
	\item Angular - Continue Progres
	\item Erlang - Dive into OTP
	\item C++ - Possible if Switching to Unreal
	\item Lua - Possible if Switching to Unreal
	\item Docker - may be useful for expanding to other students
	\item DevOps - Continue Udacity Course
	\item Networking - Continue Udacity Course
	\item Graphics - Continue Udacity Course
\end{itemize}

\end{document}
