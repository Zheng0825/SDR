%%%%%%%%%%%%%%%%%%%%%%%%%%%%%%%%%%%%%%%%%
% Masters/Doctoral Thesis 
% LaTeX Template
% Version 2.2 (21/11/15)
%
% This template has been downloaded from:
% http://www.LaTeXTemplates.com
%
% Version 2.x major modifications by:
% Vel (vel@latextemplates.com)
%
% This template is based on a template by:
% Steve Gunn (http://users.ecs.soton.ac.uk/srg/softwaretools/document/templates/)
% Sunil Patel (http://www.sunilpatel.co.uk/thesis-template/)
%
% Template license:
% CC BY-NC-SA 3.0 (http://creativecommons.org/licenses/by-nc-sa/3.0/)
%
%%%%%%%%%%%%%%%%%%%%%%%%%%%%%%%%%%%%%%%%%

%----------------------------------------------------------------------------------------
%	PACKAGES AND OTHER DOCUMENT CONFIGURATIONS
%----------------------------------------------------------------------------------------

\documentclass[
11pt, % The default document font size, options: 10pt, 11pt, 12pt
oneside, % Two side (alternating margins) for binding by default, uncomment to switch to one side
english, % ngerman for German
doublespacing, % Single line spacing, alternatives: onehalfspacing or doublespacing
%draft, % Uncomment to enable draft mode (no pictures, no links, overfull hboxes indicated)
%nolistspacing, % If the document is onehalfspacing or doublespacing, uncomment this to set spacing in lists to single
%liststotoc, % Uncomment to add the list of figures/tables/etc to the table of contents
%toctotoc, % Uncomment to add the main table of contents to the table of contents
parskip, % Uncomment to add space between paragraphs
%nohyperref, % Uncomment to not load the hyperref package
headsepline, % Uncomment to get a line under the header
]{MastersDoctoralThesis} % The class file specifying the document structure

\usepackage[utf8]{inputenc} % Required for inputting international characters
\usepackage[T1]{fontenc} % Output font encoding for international characters


\usepackage{palatino} % Use the Palatino font by default

\usepackage{hyperref}
\usepackage{listings}
\usepackage{graphicx}
\usepackage{pdfpages}
\setlength\parindent{24pt}
\graphicspath{{img/}}

%\addbibresource{example.bib} % The filename of the bibliography

%\usepackage[autostyle=true]{csquotes} % Required to generate language-dependent quotes in the bibliography

\newcommand*{\SignatureAndDate}[1]{%
    \par\noindent\makebox[2.5in]{\hrulefill} \hfill\makebox[2.0in]{\hrulefill}%
    \par\noindent\makebox[2.5in][l]{#1}      \hfill\makebox[2.0in][l]{Date}%
}%

%----------------------------------------------------------------------------------------
%	MARGIN SETTINGS
%----------------------------------------------------------------------------------------

\geometry{
	paper=letterpaper, % Change to letterpaper for US letter
	inner=2.5cm, % Inner margin
	outer=2.5cm, % Outer margin
	bindingoffset=2cm, % Binding offset
	top=2.5cm, % Top margin
	bottom=2.5cm, % Bottom margin
	%showframe,% show how the type block is set on the page
}

%----------------------------------------------------------------------------------------
%	THESIS INFORMATION
%----------------------------------------------------------------------------------------

\thesistitle{ARCAM-NET: A Software Defined Radio Network Testbed} % Your thesis title, this is used in the title and abstract, print it elsewhere with \ttitle
\supervisor{Dr. Ryan \textsc{Integlia}} % Your supervisor's name, this is used in the title page, print it elsewhere with \supname
\examiner{} % Your examiner's name, this is not currently used anywhere in the template, print it elsewhere with \examname
\degree{Masters of Engineering} % Your degree name, this is used in the title page and abstract, print it elsewhere with \degreename
\author{John \textsc{McCormack}} % Your name, this is used in the title page and abstract, print it elsewhere with \authorname
\addresses{} % Your address, this is not currently used anywhere in the template, print it elsewhere with \addressname

\subject{Electrical And Computer Engineering} % Your subject area, this is not currently used anywhere in the template, print it elsewhere with \subjectname
\keywords{} % Keywords for your thesis, this is not currently used anywhere in the template, print it elsewhere with \keywordnames
\university{\href{http://www.floridapolytechnic.org}{Florida Polytechnic University}} % Your university's name and URL, this is used in the title page and abstract, print it elsewhere with \univname
\department{\href{https://floridapolytechnic.org/academics/engineering/}{College of Engineering}} % Your department's name and URL, this is used in the title page and abstract, print it elsewhere with \deptname
\group{\href{https://floridapolytechnic.org/lab/vtc-robotics-lab/}{VTC Robotics Lab}} % Your research group's name and URL, this is used in the title page, print it elsewhere with \groupname
\faculty{\href{https://floridapolytechnic.org/staff/integlia/}{Dr. Ryan Integlia}} % Your faculty's name and URL, this is used in the title page and abstract, print it elsewhere with \facname

\hypersetup{pdftitle=\ttitle} % Set the PDF's title to your title
\hypersetup{pdfauthor=\authorname} % Set the PDF's author to your name
\hypersetup{pdfkeywords=\keywordnames} % Set the PDF's keywords to your keywords

\begin{document}

\frontmatter % Use roman page numbering style (i, ii, iii, iv...) for the pre-content pages

\pagestyle{plain} % Default to the plain heading style until the thesis style is called for the body content

%----------------------------------------------------------------------------------------
%	TITLE PAGE
%----------------------------------------------------------------------------------------

\begin{titlepage}
\begin{center}

\textsc{\LARGE \univname}\\[1.5cm] % University name
\textsc{\Large Master's Thesis}\\[0.5cm] % Thesis type

\HRule \\[0.4cm] % Horizontal line
{\huge \bfseries \ttitle}\\[0.4cm] % Thesis title
\HRule \\[1.5cm] % Horizontal line
 
\begin{minipage}{0.4\textwidth}
\begin{flushleft} \large
\emph{Author:}\\
\href{http://www.jdmccormack.com}{\authorname} % Author name - remove the \href bracket to remove the link
\end{flushleft}
\end{minipage}
\begin{minipage}{0.4\textwidth}
\begin{flushright} \large
\emph{Committee Chair:} \\
\href{https://floridapolytechnic.org/staff/integlia/}{\supname} % Supervisor name - remove the \href bracket to remove the link  
\end{flushright}
\end{minipage}\\[3cm]
 
\large \textit{A thesis submitted in partial fulfillment of the requirements\\ for the degree of \degreename}\\[0.3cm] % University requirement text
\textit{in the}\\[0.4cm]
\deptname\\[2cm] % Research group name and department name
 
\textbf{Committee Members:}\\
We \textbf{accept and approve} the thesis of the student named above.
\SignatureAndDate{Dr. Ryan Integlia}
\SignatureAndDate{Dr. Anas Salah Eddin}
\SignatureAndDate{Dr. Harish Chintakunta}


{\large \today}\\[4cm] % Date
%\includegraphics{Logo} % University/department logo - uncomment to place it


 
\vfill
\end{center}
\end{titlepage}

%----------------------------------------------------------------------------------------
%	DECLARATION PAGE
%----------------------------------------------------------------------------------------

\begin{declaration}
\addchaptertocentry{\authorshipname}

\noindent I, \authorname, declare that this thesis titled,  ``\ttitle'' and the work presented in it are my own. I confirm that:

\begin{itemize} 
\item This work was done wholly or mainly while in candidature for a research degree at this University.
\item Where I have consulted the published work of others, this is always clearly attributed.
\item Where I have quoted from the work of others, the source is always given. With the exception of such quotations, this thesis is entirely my own work.
\item I have acknowledged all main sources of help.
\end{itemize}
 
\noindent Signed:\\
\rule[0.5em]{25em}{0.5pt} % This prints a line for the signature
 
\noindent Date:\\
\rule[0.5em]{25em}{0.5pt} % This prints a line to write the date

\begin{center}
\textbf{Advisor Approval}
\end{center}

% Integlia
\SignatureAndDate{Dr. Ryan Integlia}

\end{declaration}

\cleardoublepage

%----------------------------------------------------------------------------------------
%	QUOTATION PAGE
%----------------------------------------------------------------------------------------

%\vspace*{0.2\textheight}

%\noindent\enquote{\itshape Thanks to my solid academic training, today I can write hundreds of words on virtually any topic without possessing a shred of information, which is how I got a good job in journalism.}\bigbreak

%\hfill Dave Barry

%----------------------------------------------------------------------------------------
%	ABSTRACT PAGE
%----------------------------------------------------------------------------------------

\begin{abstract}
\addchaptertocentry{\abstractname} % Add the abstract to the table of contents
\clearpage
ARCAM-Net is a Software Defined Radio Network (SDRN) testbed and platform. ARCAM-Net was born out of a need for a platform to develop SDRN applications. At the start of the project, no readily available system could be identified that would allow for user space applications to be run on an SDRN. Thus work began on the creation of ARCAM-Net. ARCAM-Net creates a fully functioning data network capable of acting as a multi-hop mesh network. Nearly every component of ARCAM-Net, except the Software Defined Radios (SDRs) themselves, is released using an open source license. The goal of ARCAM-Net is to establish a low cost platform that can be quickly implemented by anyone interested in experimenting with SDRNs. The system provides a simple user interface for interacting with the network. User space applications can be run on the network as if it was a traditional network.This document presents the network itself and also acts as a manual for working with ARCAM's first implementation. 

This work represents a first step in a longer term project to create a fully functioning Cognitive Radio Ad-Hoc Network (CRAHN). The work demonstrates the potential for using GNU Radio in conjunction with batman-adv and other Open-Mesh solutions. As the work continues forward, we hope the testbed can serve as a collaboration point between GNU Radio and Open-Mesh. Both represent next generation, open source, wireless solutions and could likely benefit from collaboration with each other. 
\end{abstract}

%----------------------------------------------------------------------------------------
%	DEDICATION
%----------------------------------------------------------------------------------------

\dedicatory{To my parents, Joe and Kathy McCormack for supporting me in all my endeavors.} 
%----------------------------------------------------------------------------------------
%	ACKNOWLEDGEMENTS
%----------------------------------------------------------------------------------------
\cleardoublepage
\begin{acknowledgements}
\addchaptertocentry{\acknowledgementname} % Add the acknowledgements to the table of contents

I would like to thank Dr. Ryan Integlia for all of his support throughout this process. I would like to further thank Dr. Anas Salah Eddin and Dr. Harish Chintakunta. I would also like to acknowledge Joseph Prine, Bradley Trowbridge, and  R. Cody Maden for all of their hard work. 

\end{acknowledgements}

%----------------------------------------------------------------------------------------
%	LIST OF CONTENTS/FIGURES/TABLES PAGES
%----------------------------------------------------------------------------------------

\tableofcontents % Prints the main table of contents

\listoffigures % Prints the list of figures

%\listoftables % Prints the list of tables

%----------------------------------------------------------------------------------------
%	ABBREVIATIONS
%----------------------------------------------------------------------------------------

\begin{abbreviations}{ll} % Include a list of abbreviations (a table of two columns)

\textbf{ADC} & \textbf{A}nalog to \textbf{D}igital \textbf{C}onverter \\
\textbf{ALFRED} & \textbf{A}lmighty \textbf{L}ightweight \textbf{F}act \textbf{R}emote \textbf{E}xchange \textbf{D}aemon\\
\textbf{BATMAN} & \textbf{B}etter \textbf{A}pproach \textbf{T}o \textbf{M}obile \textbf{A}d-hoc \textbf{N}etworking\\
\textbf{CGRAN} & \textbf{C}omprehensive \textbf{G}NU \textbf{R}adio \textbf{A}rchive \textbf{N}etwork \\
\textbf{COTS} & \textbf{C}ommerical \textbf{O}ff \textbf{T}he \textbf{S}helf \\
\textbf{CR} & \textbf{C}ognitive \textbf{R}adio\\
\textbf{CRN} & \textbf{C}ognitive \textbf{R}adio \textbf{N}etwork\\ 
\textbf{CRAHN} & \textbf{C}ognitive \textbf{R}adio \textbf{A}d-\textbf{H}oc \textbf{N}etwork\\
\textbf{DAC} & \textbf{D}igital to \textbf{A}nalog \textbf{C}onverter \\
\textbf{DIY} & \textbf{D}o \textbf{I}t \textbf{Y}ourself \\
\textbf{DSP} & \textbf{D}igital \textbf{S}ignal \textbf{P}rocessing \\
\textbf{DSS} & \textbf{D}ynamic \textbf{S}pread \textbf{S}pectrum \\
\textbf{FCC} & \textbf{F}ederal \textbf{C}ommunications \textbf{C}ommission \\
\textbf{FOSS} & \textbf{F}ree \textbf{O}pen \textbf{S}ource \textbf{S}oftware \\
\textbf{FPGA} & \textbf{F}ield \textbf{P}rogrammable \textbf{G}ate \textbf{A}rray \\
\textbf{GMSK} & \textbf{G}aussian \textbf{M}inimum-\textbf{Shift} \textbf{K}eying \\
\textbf{GRC} & \textbf{G}NU \textbf{R}adio \textbf{C}ompanion \\
\textbf{IRC} & \textbf{I}nternet \textbf{R}elay \textbf{C}hat \\
\textbf{ISM} & \textbf{I}ndustrial \textbf{S}cientific \textbf{M}edical \\
\textbf{ISP} & \textbf{I}neternet \textbf{S}ervice \textbf{P}rovider \\
\textbf{LAN} & \textbf{L}ocal \textbf{A}rea \textbf{N}etwork \\
\textbf{LNA} & \textbf{L}ow \textbf{N}oise \textbf{A}mplifier \\
\textbf{MAC} & \textbf{M}edium \textbf{A}ccess \textbf{C}ontrol \\
\textbf{MANET} & \textbf{M}obile \textbf{A}d-hoc \textbf{NET}work \\
\textbf{MPR} & \textbf{M}ulti \textbf{P}oint \textbf{R}elays \\
\textbf{OFDM} & \textbf{O}rthogonal \textbf{F}requency \textbf{D}ivision \textbf{M}ultiplexing \\
\textbf{OLSR} & \textbf{O}ptimized \textbf{L}ink \textbf{S}tate \textbf{R}outing \\
\textbf{OOT} & \textbf{O}ut \textbf{O}f \textbf{T}ree \\
\textbf{OSI} & \textbf{O}pen \textbf{S}ystems \textbf{I}nterconnection \\
\textbf{PA} & \textbf{P}ower \textbf{A}mplifier \\
\textbf{PHY} & \textbf{PHY}sical Layer \\
\textbf{PMT} & \textbf{P}oly \textbf{M}orphic \textbf{T}ree \\
\textbf{PU} & \textbf{P}rimary \textbf{U}ser \\
\textbf{RF} & \textbf{R}adio \textbf{F}requency \\
\textbf{RFNoC} & \textbf{R}adio \textbf{F}requency \textbf{N}etwork \textbf{o}n a \textbf{C}hip \\
\textbf{SCP} & \textbf{S}ecure \textbf{C}o\textbf{P}y \\
\textbf{SDR} & \textbf{S}oftware \textbf{D}efined \textbf{R}adio\\
\textbf{SDRN} & \textbf{S}oftware \textbf{D}efined \textbf{R}adio \textbf{N}etwork\\
\textbf{SSH} & \textbf{S}ecure \textbf{SH}ell \\
\textbf{SU} & \textbf{S}econdary \textbf{U}ser \\
\textbf{TAP} & Network \textbf{TAP} \\
\textbf{TCP} & \textbf{T}ransmission \textbf{C}ontrol \textbf{P}rotocol \\
\textbf{TUN} & \textbf{TUN}nel \\
\textbf{UDP} & \textbf{U}ser \textbf{D}atagram \textbf{P}rotocol \\
\textbf{UHD} & \textbf{U}SRP \textbf{H}ardware \textbf{D}river \\
\textbf{USB} & \textbf{U}niversal \textbf{S}erial \textbf{B}us \\
\textbf{USRP} & \textbf{U}niversal \textbf{S}oftware \textbf{R}adio \textbf{P}eripheral \\
\textbf{VHDL} & \textbf{V}HSIC \textbf{H}ardware \textbf{D}efinition \textbf{L}anguage \\
\textbf{VHSIC} & \textbf{V}ery \textbf{H}igh \textbf{S}peed \textbf{I}ntegrated \textbf{C}ircuits \\
\textbf{VOLK} & \textbf{V}ector \textbf{O}ptimized \textbf{L}inux \textbf{K}ernel \\
\textbf{WAN} & \textbf{W}ide \textbf{A}rea \textbf{N}etwork \\
\textbf{WLAN} & \textbf{W}ireless \textbf{L}ocal \textbf{A}rea \textbf{N}etwork \\


\end{abbreviations}

%----------------------------------------------------------------------------------------
%	PHYSICAL CONSTANTS/OTHER DEFINITIONS
%----------------------------------------------------------------------------------------

%\begin{constants}{lr@{${}={}$}l} % The list of physical constants is a three column table

% The \SI{}{} command is provided by the siunitx package, see its documentation for instructions on how to use it

%	Speed of Light & $c_{0}$ & \SI{2.99792458e8}{\meter\per\second} (exact)\\
%Constant Name & $Symbol$ & $Constant Value$ with units\\

%\end{constants}

%----------------------------------------------------------------------------------------
%	SYMBOLS
%----------------------------------------------------------------------------------------

%\begin{symbols}{lll} % Include a list of Symbols (a three column table)

%$a$ & distance & \si{\meter} \\
%$P$ & power & \si{\watt} (\si{\joule\per\second}) \\
%Symbol & Name & Unit \\

%\addlinespace % Gap to separate the Roman symbols from the Greek

%$\omega$ & angular frequency & \si{\radian} \\

%\end{symbols}



%----------------------------------------------------------------------------------------
%	THESIS CONTENT - CHAPTERS
%----------------------------------------------------------------------------------------

\mainmatter % Begin numeric (1,2,3...) page numbering

\pagestyle{thesis} % Return the page headers back to the "thesis" style

% Include the chapters of the thesis as separate files from the Chapters folder
% Uncomment the lines as you write the chapters

% Chapter 1

\chapter{Introduction} % Main chapter title

\label{Chapter1} % For referencing the chapter elsewhere, use \ref{Chapter1} 

The Advanced Radio Communication Ad-Hoc Mesh Network, or ARCAM-Net, is a platform formed through the union of several other projects and concepts \cite{selfpaper}. The main concepts embodied within ARCAM-Net are mesh networks and software defined radio networks. The current iteration of ARCAM-Net is developed using GNU Radio and Batman-adv. As work progresses on the platform, the software defined radio network can be transitioned to a cognitive radio network by leveraging the capabilities found within GNU Radio. 

%----------------------------------------------------------------------------------------

% Define some commands to keep the formatting separated from the content 
\newcommand{\keyword}[1]{\textbf{#1}}
\newcommand{\tabhead}[1]{\textbf{#1}}
\newcommand{\code}[1]{\texttt{#1}}
\newcommand{\file}[1]{\texttt{\bfseries#1}}
\newcommand{\option}[1]{\texttt{\itshape#1}}

%----------------------------------------------------------------------------------------

\section{Mesh Networks}

In a traditional Wide Area Network (WAN), a user is able to connect to the internet through an Internet Service Provide (ISP). The user will almost always pay the ISP in exchange for the ability to connect to the rest of the internet. This is considered a centralized way to connect to the internet, where the users all connect through a few central points in the ISP's infrastructure. An alternative to this type of networking is multi-hop, ad-hoc, mesh networking. Mesh networks are decentralized networks \cite{4796928}. 

In an ad-hoc network, each radio is able to communicate directly to any other radio within its transmission range. There is no need to connect to a central router \cite{4796928}. Ad-hoc networks are defined at the physical layer (PHY), or layer 1 in the Open Systems Interconnect (OSI) model. Mesh routing takes place as part of layer 2 or 3 of the OSI model depending on the routing protocol chosen. A mesh network builds upon an adhoc network by allowing radios to retransmit any packets they receive. This allows two radios to communicate over a larger distance by leveraging other radios located in between the sender and receiver\cite{0033}. 

In simple mesh networking protocols, any packet sent may flood through the network to every other radio. However, with more advanced protocols an algorithm is used to ensure that a packet follows a direct path from sender and receiver and only uses a hop if necessary. The distributed nature of a mesh network creates many unique features. The decentralized nature of a mesh network prevents issues related to single points of failure. If a node goes down, the network can reconfigure and find a new path to the target \cite{0033}. 

%----------------------------------------------------------------------------------------

\section{Software Defined Radio Networks}

Software Defined Radios (SDRs) are radio communication systems that utilize software to process radio frequency information in place of traditional hardware \cite{761033}. A radio frequency front end is able to capture and transmit signals, while the actual processing of the signal is taken care of by a digital system like general purpose processor on a traditional computer. This allows for a single piece of hardware to replace the need for multiple types of radios \cite{393001}.

A typical cell phone can have a bluetooth, wifi, gps, and cellular radio all in a very small package. In the future, these systems could be replaced by a single SDR \cite{393001}. SDRs are capable of using both digital and analog transmission protocols. They can use general purpose processors, digital signal processors \cite{393001}, or FPGAs \cite{5747366} to process the RF information. Analog to Digital Converters (ADCs) are used to receive data from the antenna while Digital to Analog Converters (DACs) are used to transmit the processed signals . 

As the name suggests, a Software Defined Radio Network (SDRN) is a network made up of SDRs. The networks can operate on a nearly infinite combination of center frequencies, amplitudes, bandwidths, and protocols \cite{7039225}. The flexibility of an SDRN is limited by the physical hardware capabilities of the SDR, the computation speed of the processing unit, and regulations from governing bodies like the Federal Communications Commission (FCC). Still, the flexibility of reconfigurable radios leads to opportunities for advancing communications infrastructure beyond traditional protocols. 

\section{Batman-adv}

The Better Approach to Mobile Ad-Hoc Networking Advanced (Batman-adv) protocol is a layer 2 mesh routing protocol developed by the Open Mesh foundation \cite{0032}. The protocol uses a Transmission Quality (TQ) metric to find a tradeoff between a low hop count and stable links. Every node on the network broadcasts an originator message to its neighbors. Each neighboring node listens for these broadcasted messages and uses the number received to create the TQ metric. Each node also broadcasts the names of its neighbors. This allows for the creation of a multihop network where packets can be forwarded through other routers until they reach their destination \cite{6115569}. 

\section{GNU Radio}

GNU Radio is an open source tool chain for digital signal processing \cite{7043470} . The GNU Radio Companion is a graphical programming tool for developing DSP systems. GNU Radio can be used for a variety of topics, but it has many tools for working with software defined radios \cite{7430125}. GNU Radio is free to use, and all of the source code is available for download under the GNU Public License \cite{6081528}. 
 
%----------------------------------------------------------------------------------------

\section{Trends Towards Cognitive Radio Environments}

Cognitive Radio Networks (CRNs) are systems of SDRs that are capable of utilizing artificial intelligence and machine learning to dynamically alter transmission patterns in real time \cite{Akyildiz2007921}. These decisions can be made by a central server that oversees all nodes on the network, but modern systems strive to make each node capable of independent decisions. 

Cognitive Radio Ad-hoc Networks (CRAHNs) combine software defined radios to form mesh networks \cite{Akyildiz2009810}. Cognitive features of the radios can allow them to change transmission parameters in accordance with link quality to ensure packets are routed properly in the mesh network. The CRs could make small changes, like increasing or decreasing their gains, in order to continue to transmit to moving nodes. They could also make larger changes, like switching entire protocols, to adapt to the needs of the network in near real time. 

Beyond ensuring good throughput in a network, CRNs also solve a major issue facing the wireless world. Frequency spectrum is a finite resource, and the available bandwidth is being quickly used up. CRNs can utilize frequency hopping to share frequency resources with traditional wireless communication systems  \cite{6892537}. A CRN can begin its operation on a specified band. If a traditional transmitter, typically called the primary user (PU), begins to operate on that frequency then the CRN will ``hop'' by switching to a different frequency and continuing operation. 

%----------------------------------------------------------------------------------------

\section{ARCAM-Net}

In order to begin work on SDRNs and CRAHNs, a testbed and research platform needed to be established. This thesis serves to describe the Advanced Radio Communication Ad-hoc Mesh Network, or ARCAM-Net, test bed and platform. The goal of this project is to create a open source, low cost, research platform that can serve as the basis for future work at Florida Polytechnic University and other organizations due to the open source nature of the project. By combining well established open source tools, ARCAM-Net could become a great tool for other research groups to begin working with SDRs. All software used by ARCAM-Net is freely available. The only costs come from the hardware. 

ARCAM-Net's architecture will be thoroughly discussed in this document, but the two major components are GNU Radio and Batman-adv. GNU Radio is an open source toolchain for creating digital signal processing tools to be used with SDRs. Batman-adv is a layer 2 mesh networking protocol. Both projects are community driven, but exist as separate entities. I hope that with ARCAM-Net the two projects can begin collaborating to help create next generation wireless network resources. 

ARCAM-Net is designed to be a platform. Therefore all of the code will be released on github. I encourage interested users to fork the repository and begin experimenting on their own. 

%----------------------------------------------------------------------------------------

% Chapter 2

\chapter{Literature Review} % Main chapter title

\label{Chapter2} % For referencing the chapter elsewhere, use \ref{Chapter1} 

%----------------------------------------------------------------------------------------

% Define some commands to keep the formatting separated from the content 
%\newcommand{\keyword}[1]{\textbf{#1}}
%\newcommand{\tabhead}[1]{\textbf{#1}}
%\newcommand{\code}[1]{\texttt{#1}}
%\newcommand{\file}[1]{\texttt{\bfseries#1}}
%\newcommand{\option}[1]{\texttt{\itshape#1}}

%----------------------------------------------------------------------------------------

\section{SDR Fundamentals}

A software defined radio, or SDR, is a radio system which performs signal processing in software instead of relying on dedicated hardware components to process signals. In a traditional hardware radio, each component is designed and built to handle a single task. If changes are needed, the hardware components must be replaced. Software defined radios are flexible and can be changed by altering code instead of components. A single SDR can be configured to act as numerous hardware radios. \cite{0018} 

An SDR itself is the hardware component which then offloads the data to a processing unit such as an FPGA, microprocessor, or general purpose computer \cite{0018}. There are many different SDRs available on the market but they share many common components. Every SDR works in slightly different ways, but they share many common components. 

An ``Ideal'' SDR only has a few stages. The first is the RF Amplifcation stage. These stages increase the gain of the signal as it enters and leaves the radio. A Low Noise Amplifier (LNA) is used on the receiver and a Power Amplifier (PA) is used on the output. The PA dramtically extends the range of the SDR and the LNA increases the signal to noise ratio of the reciever \cite{0020}. 

In the ``digital conversion'' stage, an analog to digital (ADC) and digital to analog (DAC) converter are used to change from the analog to digital domain \cite{0020}. The ADC receives the analog signal and converts it to a digital code. The DAC takes digital material and converts it to an analog signal to be transmitted. 

The final stage is the ``processing'' stage. In this stage the digital signal is sent to and from the software processor. This can come in many different forms including microcontrollers, FPGAs, and general purpose computers \cite{0020}.  

In an ideal SDR, the DAC and ADC can receive and transmit any needed signal. In practice, a real SDR requires a ``frequency converson'' stage \cite{0020}. This is an analog preprocessing stage that allows the useable frequency range of an SDR to be increased. Without this stage, the SDR's useable frequency range would be limited by the ADC and DAC to a much smaller range. 

SDR has numerous benefits compared to Traditional Radio. A single SDR can replace a multitude of radios, saving on space and cost. The operating frequency of an SDR is also flexible. This means a radio can be moved from country to country and only the software and antenna needs to be changed to operate on new frequencies \cite{0019}. As new standards, protocols, and technologies are developed, an SDR can download new software to adjust its operating parameters. Compare this to existing Wi-fi routers, where a person needs to purchase a whole new router as new standards are created \cite{0019}. 

%----------------------------------------------------------------------------------------

\section{GNU Radio and MATLAB}

There are several different tools available for creating SDR software. A few different versions are presented below. 

\subsection{GNU Radio}

GNU Radio is a free, open source, digital signal processing toolset aimed at allowing for quick and easy development of software defined radio programs. GNU Radio is licensed under the GNU General Public License (GPL) version 3 \cite{0019}. GNU Radio itself is made up of a combination of python and C++ modules. It relies heavily on the Vector-Optimized Library of Kernels (VOLK) \cite{0021} and the Boost C++ libraries \cite{0022}. 

GNU Radio programs can be developed in python or using the GNU Radio Companion \cite{0023}. GNU Radio Companion (GRC) is a graphical tool for creating signal flow graphs and can also be used to generate python code from the flow graphs. GRC is useful for creating working SDR software with minimal programming experience. It also allows for the development of parallel processing without having to understand the details of keeping everything thread safe \cite{0023}.

Custom GRC blocks can be written in C++. Flowgraphs developed in GRC can also be converted into custom blocks by first converting them to heirarchical blocks. GRC can also create interactive GUIs by leveraging Qt or Wx \cite{0023}. 

GNU Radio and GNU Radio Companion can be extended by using out of tree (OOT) modules \cite{0024}. These are a set of modules that are not part of the core feature set of GNU Radio, but have been identified as important by the community. The official OOT repository is hosted by the Comprehensive GNU Radio Archive Network (CGRAN) \cite{0025}. GNU Radio also provides the PyBOMBS tool to create an ``app store'' style interface for downloading GNU Radio tools \cite{0024}. 

GNU Radio has a large community of active users. A mailing list and internet relay chat (IRC) channel are available for asking questions. Source code is updated often and anyone can contribute to the process \cite{0003}. Though hardware is not specifically part of GNU Radio, numerous vendors such as Ettus Research and Nuand offer support libraries for working with GNU Radio \cite{0026}. 

\subsection{MATLAB}

MATLAB and Simulink are another set of tools used for working wtih software defined radios. MATLAB and Simulink are commerical options developed by Mathworks. They are not free or open soure. The two pieces of software provide similar functionality to GNU Radio, but give you the ability to leverage Mathwork's support staff and knowledge base \cite{0027}. For this project, MATLAB was evaluated but ultimately not used. The decision was made to use as many free and open-source software (FOSS) components as possible. 

\subsection{RFNoc}

Ettus Research recently released the RF Network on Chip (RFNoC) extension to GNU Radio. RFNoc allows for the creation of FPGA code blocks. Unlike normal GRC blocks, these FPGA blocks compile into code that can then be run on the FPGA located on an SDR. This can dramatically increase the speed of the block. Users can create additional RFNoC blocks in VHDL and Verilog. Unfortunately, RFNoC is only compatible with X300 and E300 series SDRs, so the software could not be used for ARCAM-Net \cite{0028}. 

%----------------------------------------------------------------------------------------

\section{Network Concepts}

A Network consists of two or more computers that are connected together in a way that allows them to share resources, transfer files, and communicate electronically. The connection between them could be physical or wireless. Traditional computer networks include local area networks (LAN) and wireless local area networks (WLAN) \cite{0029}. 

\subsection{Mesh Network}

Mesh networks, sometimes referred to as Mobile Ad-hoc NETworks(MANETs), are a type of network in which routers connect directly to nearby routers and connect indirectly to far away routers by ``hopping'' through the nearby routers until the packets reach their destination. Mesh networks can be created independent of traditional internet service providers (ISPs) as anyone with a router is able to join and create their own node on the network \cite{6908725}. 

Instead of using traditional and expensive network infrastructure, mesh networks rely on sheer number of nodes to create reliable and distributed networks. If a single node fails, the network can reconfigure and use other nodes to route packets to their destination. Expanding the network is as simple as adding more nodes to the edges of the network \cite{6908725}. 

\subsection{Mesh Network Testbeds}

The CONFINE platform uses batman-adv as the routing protocol for their mesh network testbed. However, their testbed does not utilize GNU Radio or SDRs. \cite{0001} batman-adv was also a key component of WiBed, a project to create a Commercial off-the-shelf (COTS) mesh test bed using low cost wireless routers \cite{6686492} \cite{6962154}. Though they are not specific to SDR platforms, these still show the usefulness of batman-adv as a component of a testbed. 

\subsection{GNU Radio Mesh Networks}

There has been work done in the past on establishing mesh networks using GNU Radio, however most of the research has had a different focus than ours. In \cite{4509617} and \cite{5062250} the authors use GNU Radio as a way to verify the successful use of algorithms for mesh networking. However, they do not use SDRs, instead choosing to use GNU Radio for simulation. 

The researchers in \cite{7141228} created a simple multihop test bed using three USRP radios to relay data from one computer to another. A fourth USRP acts as a primary user and attempts to block the signal. However, their work focuses on using reinforcement learning to allow for frequency hopping instead of focusing on a mesh routing protocol that would be able to scale to more radios. 

Much of the existing work done using USRPs and GNU Radio for SDR Mesh Networks revolves around implementing different parts of the Open Systems Interconnection (OSI) protocol stack from the ground up. In some papers the authors focus on the physical or mac layer \cite{5508221}. There has also been work in developing new higher layer protocols for cognitive radio mesh networks such as work done to replace Transmission Control Protocol (TCP) with a more robust protocol \cite{6686523}. These systems will usually react to frequency changes but some also change their topology based on power use \cite{6983150}.

\subsection{What is a network protocol}

A protocol is essentially a set of rules that govern the communication between two devices. A network protocol is used to define the way multiple computers can electronically send messages to one another. By standardizing protocols, computers and devices from various manufacturers can still communicate effectively \cite{0029}. 

\subsubsection{OSI Model}

The Open Systems Interconnection (OSI) Model is an abstraction that breaks network communication into seven layers \cite{0030}. Each layer has a standardized way of communicating with the layers above and below it. This means that at any given layer, a protocol can be exchanged for another without making any change to the other layers \cite{0031}. 

The first layer is the physical layer. This layers deals with the transmission and reception of unstructured raw bit streams over physical mediums. Common Layer 1 protocols include ethernet and 802.11 \cite{0030}. Layer 2 is the data linke layer. This layer passes structured data to Layer 1, and passes recognized packets on to layer 3 \cite{0031}. A common layer 2 protocol is Medium Access Control (MAC) \cite{0030}. 

Layer 3, or the network layer, deals with routing information through the network. A common layer 3 protocol is Internet Protocol version 4 (IPv4) \cite{0031}. Layer 4 is the transport layer. Layer 4 can be used to ensure that messages are delivered error free and in sequence. Common layer 4 protocols are Universal Datagram Protocol (UDP) and Transportation Controp Protocol (TCP). The next three layers are Session, Presentation, and Application layers \cite{0030}. ARCAM-Net does not make any changes to these layers so they are less pertinent to the discussion. 

Layer 5, or session layer, allows for the establishment of sessions between processes running on different stations. Layer 6, or presentation layer formats the data to be presented to the application layer. The final layer, application layer, passes the data off to the user \cite{0030}. 

\subsection{Common Mesh Network Protocols}

There are a varity of network protocols focused on the creation of Mesh Networks. Unicast routing protocols are one family of mesh network protocols. One form of these protocols is the proactice unicast routing protocol \cite{4796928}. In these networks a ``table'' is created by continuously evaluating routes to all reachable nodes. The protocol will try to keep the table as up to date and accurate as possible \cite{4796928}. When a network topology changes, the changes must be propogated throughout the network. OLSR borrows from this concept \cite{0033} \cite{5375690}. 

Another unicast method is called reactive unicast routing.These protocols are sometimes called ``on-demand'' unicast routing protocols \cite{4796928}. This type of routing finds routes only when desired by a source node. Only then does it initiate a routing discovery process. This can sometimes cause disconnect in mobile networks, but reduces the overhead associated with a proactive routing protocol \cite{0033}.

Multicast routing protocols improve on unicast methods by making networks more robust to mobile nodes that can change their location in a network. There are three main categories of multicast networks \cite{4796928}. The first is the simplest, in which packets ``flood'' the network. Every node that receives a network passes the data on to every node it is connected to. This causes an exponential growth in the number of packets on a network for every packet sent. The second is a proactive approach which precomputes paths to all nodes and then routinely updates and distributes the routing table \cite{4796928}.

In the final method, paths are created on demand. First, a source searches the mesh for the end destination. Then it computes the shortest path and sends on this path \cite{4796928}. Batman-adv is a hybrid model that borrows ideas from both multicast and unicast protocols. It is decribe as a ``biologically inspired'' network protocol \cite{0033}.  

\subsubsection{Batman-adv}

Batman-adv is a layer 2 mesh routing protocol developed by Open-Mesh. By running on layer 2, batman-adv becomes layer 3 agnostic. This allows most applications to utilize batman-adv without making any changes to the code. Batman-adv has been integrated into the linux kernel and can be found packaged with many popular distributions of linux \cite{5375690}. 

Batman-adv is a proactive routing protocol aimed at using minimal resources in order to work with embedded hardward. 
Batman-adv uses the principle that better links will provide fast, reliable connections. Every node periodically sends out a broadcast message to inform its neighbors of its existence. The neighbors then relay this information along to the rest of the network. The link quality metric is established by detecting where these broadcast packets fail \cite{5375690}. 


\subsubsection{OLSR}

The Optimised Link State Routing (OLSR) protocol uses a ``link-state'' algorithm to proactively find efficent paths between nodes. The network works by using Multi-Point Relays (MPRs) that increase network throughput by creating efficient network schemes. OLSR does this by selecting a subset of neighbouring nodes to relay data instead of every node acting as a relay. This minimies the number of packets required to establish a routing table. Network information is shared between MPRs. Every MPR has a complete routing \cite{5375690}. Batman-adv was created by Open Mesh to fix problems they saw with OLSR, namely creating an algorithm better suited to embedded system hardware \cite{0032}. 
%----------------------------------------------------------------------------------------

\section{Software Defined Radio Testbeds}

There are several well-known Software Defined Radio testbeds in use at different Universities. One major platform is the WARP platform from Rice University. This platform is made up of many custom components including the radio hardware itself \cite{7071706}. This makes the platform very expensive and limits its adaptability for use in other research facilities. 

Another platform is the Hydra platform developed at UT Austin. This platform uses GNU Radio to define PHY Layer parameters and the Click Modular Router to implement Layer 2 protocols \cite{4212821}. The Hydra platform also uses USRP radios as the hardware frontend. However, Click is an older software which, according to their website, has not had an updated release since 2011 \cite{0009}. GNU Radio has since added the Polymorphic Tree (PMT) and Message types to allow for more Layer 2 development to be done right inside GNU Radio \cite{0010}. 

The ADROIT project was another platform developed in conjunction with DARPA. This project relied heavily on Click and GNU Radio for much of its functionality. \cite{4286321}  Similarly, the University of California, Irvine and Boeing Corporation developed a testbed based on USRP Radios and GNU Radio, but they implement custom MAC layers \cite{4753441}. 

A platform similar to ARCAM-Net is presented in \cite{0002}. However, this platform uses the Optimized Link State Routing Protocol (OLSR) which has been shown to perform poorly when compared to batman-adv. OLSR is also not truly decentralized as only certain nodes relay network information \cite{5375690}

%----------------------------------------------------------------------------------------

\section{SDR and Cognitive Radios}

Cognitive Radio Networks (CRNs) are networks made up of SDRs that are capable of sensing their environment, making decisions, and changing transmission parameters \cite{Akyildiz2007921}. Essentially, Cognitive Radios are SDRs that employ aritifical intelligence and decision making to improve upon the operation of the radio. 

Cognitive Radio decision making can be made at nearly any level of the OSI model, having different effects on the overal network operation. Layer 1 decision making can be used to enable Dynamic Spectrum Sharing (DSS) \cite{5771952}. As wireless technology grows, more and more spectrum is being allocated. However, the world is quickly running out of useable spectrum. Cognitive Radios and DSS allow for cooperative sharing of available frequency \cite{5771952}. 

Primary Users (PUs), or those that have purchased a license to use the band, can operate on a given frequency \cite{4562561}. If there are any gaps in their transmission, then a secondary user (SU) can quickly transmit and receive on the frequency and then switch to a new frequency before the PU begins transmitting again \cite{4562561}. 

Layer 1 decision making can also be used to avoid congestion. If a Cognitive Radio detects there are a lot of users operating on a given frequency, then it can switch to a different channel to continue transmitting.\cite{Akyildiz2007921} 

Cognitive decision making is not limited to Layer 1. Layer 2 cognition could be used to reroute the paths used to transport packets. This could be done to improve throughput or reduce energy use \cite{6527405}. Layer 3 cognition could adjust signal gains if packets are being dropped too frequently \cite{6072038}. Layer 4 could detect when packets are going directly to a single node, and adjust the nodes frequency so they have their own independent channel of operation \cite{5062176}. In general, Cognitive Radios improve upon SDRs in that they can react to and predict changes in order to refine the operation of the network. 


  
%----------------------------------------------------------------------------------------

\section{State of the Practice}

\subsection{RTL-SDR}

	The RTL-SDR is an SDR tool discovered by the DIY and hacking community. Its original purpose is to be used as a digital TV tuner. However, it was discovered that this system could also be used to general SDR purposes. There is now a large community dedicated to using this tiny SDR to recieve various different signals. Prior to the creation of the open source drivers for the RTL-SDR, the most popular devices for SDR came from USRP. The USRP devices are fantastic products, but cost at least \$1,000 and can cost quite a bit more with additional features. The RTL-SDR is based on rhe Realtek RTL2832U chip. This device can often be purchased for between \$20 and \$30 \cite{6526525}. The range for the RTL-SDR is typically 64 MHz	to 1700 MHz, however this varies depending on which tuner the manufacturer paired with it. The authors in \cite{6526525} paired the RTL-SDR with a mixer in order to lower the range all the way to DC. For this, they used the NE6062AN chip.  
	
	Starting with Release 2013b, MATLAB/Simulink now have a support package that targets RTL-SDR devices. In Simulink, the package contains a single block called ``RTL-SDR Receiver.'' This block allows the user to tune the center frequency, change the tuner gain, set the sampling rate, and alter the frequency correction factor. The block can then output the complex envelope (IQ) of the recieved signal in both floating point and integer formats\cite{6893337}.  Due to the open nature and low cost of the RTL-SDR, the authors in \cite{6821718} propose using this as a tool set for teaching DSP and Communications principles to students. 
	
	When the cost of the system is far less than that of a
	textbook, it is easy to see why this could become a valuable learning tool for many students. UC Berkley has already begun to use the RTL-SDR as one of the project assignments in their digital signal processing course. 	There have been efforts to use the RTL-SDR with the popular Raspberry Pi computing platform. However, at least with the B+ model there is not enough power available to process the signal. Instead, it has to be used as a TCP server that is then able to forward the data on to a more powerful computer\cite{6938691}. Currently, it does not appear as though anyone has tested this with the Raspberry Pi 2 microcomputer. Other work has been done to estimate the cost savings of using a USRP in conjunction with several RTL-SDRs to replace existing DSP lab infrastructure \cite{6726630}. 
	
	The most unfortunate downside to using the RTL-SDR is that it is unable to transmist. However, researchers in \cite{6922233} proposed a system in which a USRP SDR is used as a master device that broadcasts out to a series of slave nodes that can only listen for information. In their system they used a tool called GStreamer to pass video data into GNURadio. This information was then broadcast to multiple computers in a room running an RTL-SDR with GNURadio. The nodes were all able to view the video stream in near real time.  

\subsection{USRP}

ARCAM-Net uses the Universal Software Radio Peripheral (USRP) created by Ettus Research, a division of National Instruments\cite{0006}. Ettus also released the Universal Hardware Drivers (UHD) \cite{0007} which allow for the use of the USRP with GNU Radio. USRP radios are supported by both Mathworks \cite{0027} and GNU Radio \cite{0026}. Ettus research provides the open source USRP Hardware Drivers (UHD) to allow the devices to interface into various tools \cite{0007}.

The USRP line of radios are able to transmit and recieve data. The B200 and B210 operate over USB 3.0 \cite{0034}. The E310 radios are standalone systems with their own onboard processor built into the FPGA \cite{0035}. The newer E310's also come with an onboard battery, allowing them to operate in a mobile fashion \cite{0036}. The USRP line of radios is used heavily in academia. The radios form the basis of numerous testbeds  \cite{4753441} \cite{4212821}. They are also used to validate data from simulations \cite{5508221}. 



%----------------------------------------------------------------------------------------

\section{State of the Art}

\subsection{Cognitive Radio Networks}

Cognitive Radio Networks represent a technological evolution from Software Defined Radio Networks \cite{6599059}. In a traditional Software Defined Radio Network the waveforms and transmission protocols are flexible. However, they are predefined by the software loaded onto the computer or radio. To change operating parameters, a different program is run or the user manually alters transmission patterns. In a Cognitive Radio Network, changes can be made intelligently through the use of machine learning and artificial intelligence. 

A major benefit of Cognitive Radio Networks is that they can use Dynamic Spectrum Allocation (DSA) \cite{6892537}. Current wireless transmission is performed on static, licensed bands. The spectrum must be purchased through the Federal Communications Commission (FCC) and the user can then only operate on that particular frequency. With Dynamic Spectrum Allocation, radios are able to find bands of frequency that are not currently being utilized and perform there transmissions there. Once a licensed user begins transmitting on that frequency, the radios can shift to a new band and continue operation. Dynamic Spectrum Allocation helps to allievate the problems of spectrum sharing and drastically increases the useability of each frequency band \cite{6599059}. 

As with SDRs, Cognitive Radios are able to alter operating parameters such as transmission power, frequency, modulation type. Unlike SDRs, any of these parameters can be changed through the use of Artifical Intelligence and Machine Learning. The general cognitive functions of CRNs occur in what is called a ``cognitive cycle.'' \cite{5639025} This cycle is made up of four componenets: Sensing, Analysis, Reasoning, and Adaptation. In spectrum sensing and analysis, the radios are constantly scanning over a given frequency band, looking for openings. In the reasoning phase, the radio looks at the data for each frequency band and quantifies the ``ideal'' band to begin transmitting on. In the adaptation phase the radio switches to the new operating parameters \cite{5639025}. 

Each radio may be capable of performing these actions on their own, or they may leverage a spectrum broker. The spectrum broker is a centralized cognitive radio that may have wider band sensing capabilities \cite{5639025}. This central broker will then send commands to the other cognitive radios to dictate which frequency they should continue to transmit on. This decentralized approach is less ideal, as it limits the scalability of the network. However, it can often be less computationally complex to deploy \cite{5639025}. 


\subsection{Biologically Inspired Protocols}

These protocols are based on algorithms that are designed to mimic naturally occuring phenomenon. This leads to novel and efficient algorithms well suited for the distributed nature of a mesh network.

\subsubsection{Swarm Protocols}

Swarm Intelligence is a class of algorithms that are based around the social behaviors of insects. They are highly applicable to mesh networks as they are designed for distributed and decentralized problem solving. Adaptive Swarm-based Distributed Routing (Adaptive-SDR) is one type of swarm routing protocol that can be used for SDR applications. In this protocol, nodes are divided into colonies. Each node learns how to get to the other colonies, and how to get to nodes within their own colonies. When a packet traverses the network, each node forwards the data on the path to the correct colony first. Then nodes within the colony forward on to the correct node \cite{1005496}. This protocol is similar to the one used by Batman-adv \cite{0033}.


\subsubsection{Hive Protocols}

Hive protocols emulate the behavior of bees that forage and return to a central hive \cite{6863008}. In this network structure, the network is divided into several foraging zones. To find a path to a node, a bee (in the form of a packet) will seek out a destination node. Once the packet reaches the destination, another one will be sent back to the source on the same route. When it reaches the source this packet will then communicate the distance, direction, and quality of the path between the two nodes. If multiple bees are sent out, they will then be able to communicate the optimum route to take for future packets \cite{7373160}. 

\subsubsection{Ant Colony Optimization Protocols}

Ant Colony Optimization (ACO) Protocols are modeled off of ants cooperating to find food for their colony \cite{7030885}. In this protocol, a set of ``Ants'' are sent out in the form of discovery packets. The first set try to find the source node within the network. The other set try to find the destination node. As the ants walk, they deposit pheromones within the network to mark the nodes they have traversed. As more ants visit a given node, the quantity of pheromones at that location will begin to increase. After awhile, the optimal path between a source and destination node can be determined by following the path with the largest amount of pheromones \cite{7030885}. 

\section{Decision Making and Biologically Inspired Decision Making}

At a very simple level, decision making can just be automatic gain correction (AGC). A cognitive radio with AGC may adjust its gain to its highest level until it begins to receive data. At the highest gain level, it may detect clipping and begin to lower its gain until an undistorted signal, with a good signal to noise ratio, is acheived. \cite{6497193}

Genetic Algorithms can be applied to MIMO cognitive radio systems. In these CR environments, a genetic algorithm is used to optimize bandwidth, band efficiency, transmission power, data rate, and bit error rate \cite{7124804}. The genetic algorithm works by using a fitness score to indicate the quality of the established link. Combinations of parameters that generate high fitness scores are allowed to ``breed'' into new generations of parameters. Eventually, an optimal set of parameters is discovered and the Cognitive Radio continues to transmit and receive at these settings \cite{7124804}.

Biologically-inspired algorithms are those that mimic nature in their execution. Biologically-inspired spectrum sharing is a major component of Cognitive Radio Networks. One such algorithm is called ``BIOlogically-Inspired Spectrum Sharing'' or the BIOSS algorithm \cite{4224259} \cite{5686503}. This algorithm uses a task allocation model based on insect colonies. The various parameters that an SDR can monitor are given response thresholds. If this threshold is passed then the radio will react and change to a new parameter. For example, if a large enough primary user began to use the spectrum the cognitive radio is transmitting on then it will switch to a new frequency \cite{4224259} \cite{5686503}. 

Another biologically inspired algorithm is used to form consensus-based spectrum sensing \cite{5464224}. This algorithm is based off of flocks of birds and group decision making of other types of animals. In this scenario, each node senses the spectrum around itself. When a given node begins to sense a primary user, it lets the nodes adjacent to it know \cite{5464224}. These nodes then propagate it further. Overtime, the entire network will ``steer'' towards a new frequency the same way a school of fish may collectively steer away from a predator. 


The Ant Colony Optimization (ACO) algorithm has also been applied to dynamic spectrum access in cognitive radio networks \cite{5600815} Each node explores possible bands that it can transmit in and then uses a simulated pheromone to indicate good locations to transmit on. These pheromones have a decay time to ensure that they are constantly updated. Once a band has a large number of nodes releasing pheromones on it, then the other nodes will converge on this frequency. This process can be repeated continuously to move away from primary nodes as they appear \cite{5600815}. 


%%----------------------------------------------------

% TODO

%%----------------------------------------------------


\subsection{Cognitive Radio Networks}

Cognitive Radio Networks represent a technological evolution from Software Defined Radio Networks \cite{6599059}. In a traditional Software Defined Radio Network the waveforms and transmission protocols are flexible. However, they are predefined by the software loaded onto the computer or radio. To change operating parameters, a different program is run or the user manually alters transmission patterns. In a Cognitive Radio Network, changes can be made intelligently through the use of machine learning and artificial intelligence. 

A major benefit of Cognitive Radio Networks is that they can use Dynamic Spectrum Allocation (DSA) \cite{6892537}. Current wireless transmission is performed on static, licensed bands. The spectrum must be purchased through the Federal Communications Commission (FCC) and the user can then only operate on that particular frequency. With Dynamic Spectrum Allocation, radios are able to find bands of frequency that are not currently being utilized and perform there transmissions there. Once a licensed user begins transmitting on that frequency, the radios can shift to a new band and continue operation. Dynamic Spectrum Allocation helps to allievate the problems of spectrum sharing and drastically increases the useability of each frequency band \cite{6599059}. 

As with SDRs, Cognitive Radios are able to alter operating parameters such as transmission power, frequency, modulation type. Unlike SDRs, any of these parameters can be changed through the use of Artifical Intelligence and Machine Learning. The general cognitive functions of CRNs occur in what is called a ``cognitive cycle.'' \cite{5639025} This cycle is made up of four componenets: Sensing, Analysis, Reasoning, and Adaptation. In spectrum sensing and analysis, the radios are constantly scanning over a given frequency band, looking for openings. In the reasoning phase, the radio looks at the data for each frequency band and quantifies the ``ideal'' band to begin transmitting on. In the adaptation phase the radio switches to the new operating parameters \cite{5639025}. 

Each radio may be capable of performing these actions on their own, or they may leverage a spectrum broker. The spectrum broker is a centralized cognitive radio that may have wider band sensing capabilities \cite{5639025}. This central broker will then send commands to the other cognitive radios to dictate which frequency they should continue to transmit on. This decentralized approach is less ideal, as it limits the scalability of the network. However, it can often be less computationally complex to deploy \cite{5639025}. 



\subsection{Cognitive Networks}

\subsection{Cognitive Radio Ad-Hoc Networks}

Cognitive Radio Ad-Hoc Networks (CRAHNs) are essentially mesh networks made up of intelligent software defined radios \cite{Akyildiz2009810}. They are a specialized subset of Cognitive Radio Networks \cite{5639025}. CRAHNs seek to combine the flexibility of Cognitve Radios with the distributed network routing of mesh networks. CRAHNs allow for dynamic route topology due to the ability of ad-hoc networks to connect to any and all peers within their broadcast range. The features of the radio, such as frequency and amplitude, can be adjusted to force the topology of the users choosing. This can also be done dynamically using artifical intelligence and decision making \cite{Akyildiz2009810}. 

Routing in multi-hop CRNs or CRAHNs has to deal with a number of challenges. The dynamic changing of the available spectrum can alter the topology of the network as different frequencies react to the environment in different ways. The routing metrics for CRAHNs will need to be able to adjust each time a hop is performed in order to rebuild the topology \cite{6599059}. The routing protocol will also have to be able to handle the transmission delay that will occur as the nodes are changing frequency \cite{6599059}. During the change, the network will be unreachable. The routing protocol will need to be able to begin transmission again just as the network comes back online or risk dropping packets. 


\section{Software Defined Networks}

\section{SDRN vs SDN}

\section{802.11s}

\section{Industry State of the Practice}

General Dynamics produces the S-Band software defined radio. This radio is used in the Space Communications and Navigation Testbed on the Internation Space Station. This SDR provides experimenters an opportunity to develop and demonstrate experimental waveforms in space \cite{6497193}. 
Mention the stuff Harris uses here.

SDRN and CRN devices are ideal for emergency response and military operations. Various countries operate using different standards for radio transmission. SDRs and CRs allow for better coordination, as the radios can be configured to operate using a form not commonly used by a given organization. In emergency response, the parameters of the radio can also be changed to provide additional bandwidth and spectrum for the duration of the operation \cite{5639025}. 

%http://ieeexplore.ieee.org.flpoly-proxy.flvc.org/xpl/articleDetails.jsp?arnumber=5747366&newsearch=true&queryText=Harris%20Corporation%20SDR

%----------------------------------------------------------------------------------------
 
% Chapter 3

\chapter{Methodology} % Main chapter title

\label{Chapter3} % For referencing the chapter elsewhere, use \ref{Chapter1} 

%----------------------------------------------------------------------------------------

% Define some commands to keep the formatting separated from the content 
%\newcommand{\keyword}[1]{\textbf{#1}}
%\newcommand{\tabhead}[1]{\textbf{#1}}
%\newcommand{\code}[1]{\texttt{#1}}
%\newcommand{\file}[1]{\texttt{\bfseries#1}}
%\newcommand{\option}[1]{\texttt{\itshape#1}}

%----------------------------------------------------------------------------------------

\section{Design}

\subsection{SDR}

\subsection{SDR Network/SDN}

\subsection{Intelligence and Automatic Decision Making Components}

\subsubsection{Sensors}

\subsubsection{HMI}

\subsubsection{Network}

\subsubsection{Application: Security}

\subsection{Sensor Integration}

\subsubsection{Groups of Sensors}

\subsubsection{Imagers}

\subsubsection{Audio}

\subsubsection{LIDAR}

\subsection{Human Machine Interfaces}

\subsubsection{User Data}

\subsection{Augmented Reality}

%----------------------------------------------------------------------------------------

\section{Fabrication and Build Process}

\subsection{SDR}

\subsection{SDR Networks and SDN}

\subsection{Intelligence}

\subsubsection{Sensors}

\subsubsection{HMI}

\subsubsection{Network}

\subsubsection{Application:Security}

\subsection{Sensor Integration}

\subsubsection{Groups of Sensors}

\subsubsection{Imagers}

\subsubsection{Audio}

\subsubsection{LIDAR}

\subsection{HMI}

\subsection{AR}

%----------------------------------------------------------------------------------------

\section{Test - SDR (Environment/Setup)}

\subsection{Signal Strength}
\subsection{Range}
\subsection{Noise}
\subsection{Loss}
\subsection{Error (inaccuracy)}
\subsection{Energy Use}
\subsection{Spectrum}
\subsection{Protocol}


%----------------------------------------------------------------------------------------

\section{Test - Network}

\subsection{Signal Strength}
\subsection{Range}
\subsection{Noise}
\subsection{Loss}
\subsection{Error}
\subsection{Energy Efficiency}
\subsection{Latency}
\subsection{Throughput}
\subsection{Topology Effects}
\subsection{Protocol}
\subsection{Interference}
\subsection{Crowding}
\subsection{Crosstalk}

%----------------------------------------------------------------------------------------

\section{Test - AR}

\subsection{Signal Strength}
\subsection{Noise}
\subsection{Loss}
\subsection{Error}
\subsection{Topology Effects}
\subsection{Life Cycle}

%----------------------------------------------------------------------------------------

% Chapter 4

\chapter{Findings} % Main chapter title

\label{Chapter4} % For referencing the chapter elsewhere, use \ref{Chapter1} 

%----------------------------------------------------------------------------------------

% Define some commands to keep the formatting separated from the content 
%\newcommand{\keyword}[1]{\textbf{#1}}
%\newcommand{\tabhead}[1]{\textbf{#1}}
%\newcommand{\code}[1]{\texttt{#1}}
%\newcommand{\file}[1]{\texttt{\bfseries#1}}
%\newcommand{\option}[1]{\texttt{\itshape#1}}

%----------------------------------------------------------------------------------------

\section{Results}

\subsection{Network Benchmarks}

\begin{figure}
	\centering
	\includegraphics[scale=0.5]{902sheet}
	\caption{The data received from operating at 902/915 MHz.}
	\label{fig:902}
\end{figure}

\begin{figure}
	\centering
	\includegraphics[scale=0.5]{2400}
	\caption{The data received from operating at 2.4/2.5 GHz.}
	\label{fig:2400}
\end{figure}

\begin{figure}
	\centering
	\includegraphics[scale=0.5]{alldata}
	\caption{The Averages from all three tests.}
	\label{fig:alldata}
\end{figure}

The results of the Network Benchmark tests from section 4 part A are summarized in Figure \ref{fig:alldata}. In all cases, the single point to point communication, without the batman-adv protocol running, resulted in a much lower packet loss. This served as our control group. However, in the two sets of lower frequency ratings, the packet loss remained below 50\%. Also, the increase in time as hops were added has a roughly linear change. This means the overhead of adding more hops is not unmanageable. A full listing of tests run for the 902/915 MHz, and 2.4/2.5 GHz cases are provided in Figure \ref{fig:902} and Figure \ref{fig:2400} respectively. These tables show that running at the higher frequencies causes the SDRN to drop a lot more packets, especially when moving through the full four hops.  


\subsection{Route Changes}

\begin{figure}
	\centering
	\includegraphics[scale=0.4]{2PotentialHops}
	\caption{The initial condition from the test presented in Section 4 B, where there are two possible routes the packet can take. The output is from running batctl. The link quality is listed under the column labeld (\#/255). A higher number represents a better quality connection.}
	\label{fig:2Hops}
\end{figure}

\begin{figure}
	\centering
	\includegraphics[scale=0.4]{hopchange}
	\caption{After the gain is reduced in the test presented in Section 4 B, the packets are now routing through a different node.}
	\label{fig:NewHop}
\end{figure}

The route changing feature of batman-adv showed success with our test setup from section 4 part B. Initially, batctl reported two possible links, one with a link quality of 55, and the other with a link quality of 18. As we decreased the gain of the intermediate node, the link quality reported by batctl also decreased. Eventually, batman-adv switched and began using the other node. At this point, it no longer saw the original node, and reported a link quality of 75 on the alternate one. The initial setup can be seen in Figure \ref{fig:2Hops}. After the change, the routing table appeared as it does in Figure \ref{fig:NewHop}. This feature works in the SDR system, and can continue to be used without significant changes.  

\subsection{Frequency Changes}

\begin{figure}
	\centering
	\includegraphics[scale=0.4]{FrequencyShift922-920}
	\caption{The result of using ALFRED to shift frequencies. One node is left behind as the others move to the new channel. Image collected using the Tektronix RSA306 Spectrum Analyzer}
	\label{fig:freqshift}
\end{figure}


Using A.L.F.R.E.D. to distribute frequency hopping showed mixed results. Using the setup from section 4 part C, we were able to get the nodes to change frequency in unison, but not reliably. A.L.F.R.E.D. itself is designed for a traditional Wi-Fi environment, and therefore does not have an expectation that the other nodes will become completely unreachable due to changes in operating frequency \cite{0015}. We were finding that some nodes would switch before A.L.F.R.E.D. had propagated the data table to the other nodes. This would leave one node with an out-of-date table, meaning it would not make the frequency change. In the current iteration of the project, there is no way for these orphaned nodes to find the rest of the network again. Figure \ref{fig:freqshift} shows a situation in which four out of five nodes were able to make the jump, with one node remaining at the original frequency.


%----------------------------------------------------------------------------------------

\section{Data Analysis}

In an ideal situation, we would see minimal packet loss despite introducing batman-adv and multihopping into the network. Additional tests will need to be run to discover what causes the 1 hop with batman-adv to drop many more packets than 1 hop without batman-adv. It is possible the packets are dropped as routing table information is shared, but this has not been confirmed yet. 

It is important to notice that packet loss and round trip time of packets increases lineraly with each hop. This is good, as it means each hop adds a finite amount of delay and packet loss. It would be unlikely that the delay would be constant when adding more hops, so a linear time increase is good. 

Another interesting issue with this test was that we had to stagger the operating frequencies to force the straight line topology we were looking for. Batman-adv worked very well at finding alternative paths to each node in the network. However, if the tests were left running for long periods of time it was possible the network would have been able to reconfigure into a topology other than a straight line chain. Therefore, it was necessary to manually ensure that the network would stay in the desired configuration in order to maintain a valid test. 

The route changing feature of batman-adv worked fine with the SDR hardware. Initially, there was uncertantity if batman-adv would be able to compute a proper metric for the link quality of each node. This link quality metric is what batman-adv uses to determine the best ``next hop'' to reach a different node. However, our tests showed the routing table could be properly built and updated as link quality decreased.

Our tests on A.L.F.R.E.D. showed that even though it was possible to exchange frequency information using A.L.F.R.E.D., it was not an ideal system. There was no way to ensure A.L.F.R.E.D. would pass the data to every node in the network. Therefore, the only possible way to improve upon this system would be to wait longer periods of time before shifting frequencies. If this was being used to avoid a primary user, a long delay would not be acceptable as it would interfere with the PU's use of the spectrum. Therefore, a new implementation of this feature will need to be produced. 



%----------------------------------------------------------------------------------------

\section{Assumptions and Discussion of Error}

One assumption is that the straight line configuration shown in Figure \ref{fig:HopMess} reflects how the network would behave in a full scale environment. Here we are using this configuration to determine the overhead associated with each additional hop. However, in a more robust network configuration it is likely there will be more than one path to any given node. These additional paths may be necessary to lower the overall packet loss of the network. It may be more appropriate to test this type of configuration in simulation rather than in the real world. 

We also did not test these setups in an anechoic chamber. Therefore, it is not possible to know how reflections influenced our data set. Reflections of RF waveforms can cause both constructive and destructive interference. This means that the reflections could either improve or reduce the signal quality at any given node.  

A uniform operating pattern in each node and in each antenna is assumed. Without the necessary equipment, it is impossible to truly characterize the antennas or transmission patterns of the SDRs. This is not likely to be a source of error based upon the information in the data sheets, but it is something to be aware of. 

%----------------------------------------------------------------------------------------

\section{Error Bars}

%----------------------------------------------------------------------------------------
 
% Chapter 5

\chapter{Discussion} % Main chapter title

\label{Chapter5} % For referencing the chapter elsewhere, use \ref{Chapter1} 

%----------------------------------------------------------------------------------------

% Define some commands to keep the formatting separated from the content 
%\newcommand{\keyword}[1]{\textbf{#1}}
%\newcommand{\tabhead}[1]{\textbf{#1}}
%\newcommand{\code}[1]{\texttt{#1}}
%\newcommand{\file}[1]{\texttt{\bfseries#1}}
%\newcommand{\option}[1]{\texttt{\itshape#1}}

%----------------------------------------------------------------------------------------

\section{Constraints and Limitations}

%----------------------------------------------------------------------------------------

\section{SDR/SDN In Unity}

%----------------------------------------------------------------------------------------

\section{SDR/SDN Controllability and Feedback}

%----------------------------------------------------------------------------------------

\section{SDR Network}

%----------------------------------------------------------------------------------------

\section{Cognitive Networks}

%----------------------------------------------------------------------------------------

\section{Software Defined Environment}

%----------------------------------------------------------------------------------------

\section{Human Machine Interaction}

%----------------------------------------------------------------------------------------

\section{Applications}

\subsection{SDR Network AR}

\subsection{Security}

\subsection{Gaming and Crowdsourcing}

%----------------------------------------------------------------------------------------
 
% Chapter 6

\chapter{Conclusion} % Main chapter title

\label{Chapter6} % For referencing the chapter elsewhere, use \ref{Chapter1} 

%----------------------------------------------------------------------------------------

% Define some commands to keep the formatting separated from the content 
%\newcommand{\keyword}[1]{\textbf{#1}}
%\newcommand{\tabhead}[1]{\textbf{#1}}
%\newcommand{\code}[1]{\texttt{#1}}
%\newcommand{\file}[1]{\texttt{\bfseries#1}}
%\newcommand{\option}[1]{\texttt{\itshape#1}}

%----------------------------------------------------------------------------------------

\section{Summary}

ARCAM-Net allows for researchers to create a fully functioning SDRN in a short period of time. In it's current form, ARCAM-Net's auotmated processes could scale to up to 255 simultaneous nodes. More nodes could be added with only a few changes to the start scripts. ARCAM-Net is designed to bridge the Open-Mesh and GNU Radio projects while providing researchers with the tools they need to design, experiment, and test. 

%----------------------------------------------------------------------------------------

\section{Conclusion}

ARCAM-Net is a Software Defined Radio Network (SDRN) testbed and platform. ARCAM-Net was born out of a need for a platform to develop SDRN applications. At the start of the project, no readily available system could be identified that would allow for user space applications to be run on an SDRN. Thus work began on the creation of ARCAM-Net. ARCAM-Net creates a fully functioning data network capable of acting as a multi-hop mesh network.

Both Batman-adv and GNU Radio represent next generation, open source, wireless solutions and each organization could likely benefit from collaboration between the two projects. This work can be used as an excellent starting point for anyone looking to get an SDRN up and running quickly to begin prototyping other sections of the tool chain. We plan on releasing the code as well as a handful of tools to help other researchers get started. This work can serve as the first step in creating an equivalent Cognitive Radio Network platform and testbed. 

This work represents a first step in a longer term project to create a fully functioning SDRN. The work demonstrates the potential for using GNU Radio in conjunction with batman-adv and other Open-Mesh solutions. As the work continues forward, we intend for the testbed to serve as a collaboration point between GNU Radio and Open-Mesh. 

We were able to show that packets can successfully traverse the network. Packet loss increased linearly with each additional node the packet traversed. Initially, packet loss was near 95\%, but was brought down with changes to the flowgraphs. We were also able to prove batman-adv's multihop and rerouting features worked properly in the Software Defined Radio Network. The attempts to use A.L.F.R.E.D. as a means of sharing frequency data proved to have issues. The information could be passed from node to node, but would not reliably reach all nodes on the network. 

Several applications were run over the network. Secure Shell was used to run commands remotely on another computer. Secure Copy was also used to transfer files from one machine to another. Both of these commands would work with more than one hop in between the nodes. A chat application was developed in Python using UDP packets. Nodes could broadcast text messages to the other machines on the network using the broadcast address. Finally a short, distributed program was developed in Erlang. This program was able to send commands to a process on a different machine, over the Software Defined Radio Network. 

Several tools were also generated for use with the network. The Flask Server automatically setup the GNU Radio Flowgraph and then configured the batman-adv network on that node. Additonally, a script was generated to allow for all nodes on the network to be setup with just a single command. Finally, an install script was generated that could create either a full install or a virtual machine. This would install all dependencies and then configure GNU Radio and Batman-adv. 



%----------------------------------------------------------------------------------------

\section{Recommendation}

ARCAM-Net should reused and extended upon by future research efforts. It is important to continue to document and project and ensure easy access to the documentation to outside parties. As with any platform, the value is in numerous groups working together on a standard set of tools. 

Moving forward, the GMSK blocks can be replaced with different protocols. Having the ability to switch between different protocols, such as OFDM, would further show the flexibility of ARCAM-Net. The same is true of batman-adv. Integrating  different protocols or including an SDN tool such as OpenFlow would make ARCAM-Net more extensible and allow control to continue to the upper layers. Finally, allowing the network to be remotely accessible would allow for other researchers to utilize the platform even if they lack equipment of their own. 
%----------------------------------------------------------------------------------------

\section{Future Work}

There are many areas of this project that can serve as starting points for new projects. First, A.L.F.R.E.D. needs to be replaced by something more robust. A.L.F.R.E.D. could continue to be used to store distributed data that will not change often, but should not be used to store frequency data. Instead, a new tool will need to be create that can safely propagate data to other nodes, and wait for an acknolwedgement of the received information. This tool will be useful in creating the next major revision to ARCAM-Net. 

As stated previously, ARCAM-Net is a SDRN, not a CRN or CRAHN. In order to move from an SDRN to a cognitive radio environment, it will be necessary to begin to implement artifical intelligence and decision making in some form. The decision making could be split Layer 1 and Layer 3 decision making. On Layer 1, DSP would be employed to look for PU's, noise and distortion on incoming packets, overcrowded networks, and other issues. Once these problems are deteced the system will then switch to a new frequency. On Layer 3 the nodes will look for excessive dropped packets, poor batman-adv link quality, orphaned nodes, and other factors. The system will then adjust gains and frequencies until this layer has been optimized. 

While running experiments, batctl was used to monitor the quality of each node. Though batctl works fine for a few nodes, as the network scales it would make finding issues and bottle necks more difficult. Another great way to improve upon the current project would be to implement a more interactive environment for monitoring the network. A central ``control'' server could be created to monitor all of the nodes on the network and display information in a 2D or 3D environment. This would make network monitoring more organized, as batctl produces different data on each node and therefore must be monitored on many computers simultaneously. 
%----------------------------------------------------------------------------------------


%----------------------------------------------------------------------------------------
%	BIBLIOGRAPHY
%----------------------------------------------------------------------------------------

%\printbibliography[heading=bibintoc]

%----------------------------------------------------------------------------------------
\bibliographystyle{IEEEtran}
\bibliography{example}


%----------------------------------------------------------------------------------------
%	THESIS CONTENT - APPENDICES
%----------------------------------------------------------------------------------------

\appendix % Cue to tell LaTeX that the following "chapters" are Appendices

% Include the appendices of the thesis as separate files from the Appendices folder
% Uncomment the lines as you write the Appendices

% Appendix A

\chapter{Running ARCAM-Net} % Main appendix title

\label{AppendixA} % For referencing this appendix elsewhere, use \ref{AppendixA}

This appendix will serve as a ``How-to'' manual for getting started with ARCAM-Net. 

\section{Configuring the Computer}

First, Ubuntu 14.04 LTS will need to be installed on each node computer. Installing Ubuntu is beyond the scope of this document. However, there are plenty of easy to follow tutorials available online. 

Once installed, the remaining steps can be done manually, or you can use a shell script I created to automate most of the process. This shell script is available \href{https://github.com/jmccormack200/GnuRadio-Vagrant-Script}{on Github}. As the name of the repository suggests, you can also use this to setup and configure a virtual machine to run the SDR software using Vagrant and Virtual Box. If you are unable to install an OS on the computers you are using, this is a good alternative. Please be aware that both options involve a lot of downloading and compiling of software. This process can take up to 8 hours so be sure to have plenty of time to let this run. An internet connection is required. 

\subsection{Installing Natively}

\code{git clone https://github.com/jmccormack200/GnuRadio-Vagrant-Script.git}
\code{cd GnuRadio-Vagrant-Script}
\code{sudo sh bootstrap.sh}


\subsection{Installing Through Vagrant}

Download and install Vagrant and Virtualbox to your computer. Then run:

\code{vagrant up}

The rest should be configured automatically.

\subsection{Manual Install}

If you prefer to do things yourself just be sure to follow along with the ``bootstrap.sh'' file below:

\lstinputlisting[language=bash, breaklines=true, caption={Installer Shell Script}]{bootstrap.sh}

\section{Getting The Flowgraphs}

The plubic version of ARCAM-Net can be found \href{https://github.com/jmccormack200/ARCAM-Net-Public.git}{on Github}. The public version will be the most stable release of the network, and will be updated each time the private repository is used to publish a new paper. To get the flowgraphs use: \\

\code{git clone https://github.com/jmccormack200/ARCAM-Net-Public.git}
\code{cd ARCAM-Net-Public}
\code{cd Flowgraphs}

\section{Running The Flowgraphs}

At the time of this writing, there are four files in the Flowgraph folder of the repository. The four files are actually two sets of files. The files ending in ``.grc'' are the GNU Radio Companion implementations and the files ending in ``.py'' are the same files but written in python. The Python files are automatically generated from the GRC blocks. The files ``batNoGui.grc'' and ``batmanNoGui.py'' represent the files used by ARCAM-Net which do not contain a GUI. This is needed to run the flowgraph in a background thread while using the Flask web server. If you just plan on running the flowgraphs I recommend using ``broadcastwithFreqNoMac.grc'' or ``broadcastwithFreqNoMac.py'' as this will also include a helpful GUI for interacting with the flowgraph.

These flowgraphs are used by ARCAM-Net but are not the entire network. They are just the scripts used to control the SDR. 

To run a GNU Radio script open it using: \\

\code{sudo gnuradio-companion Name\_of\_Script} \\

Then at the top of the flowgraph click ``build'' and then ``Run''. Alternatively, they can be run straight from the command line using: \\

\code{sudo python Name\_of\_Script} \\

Please note that admin privileges are needed in order to create the TUN/TAP device. If running the python file from the command line, it is possible to use command line arguments to adjust certain parameters prior to running. We found this to be useful as we could start the network a specific gain instead of having to go through and manually change the gains at each node. 

\subsection{Flow Graph to Mesh Network}

Running just the flowgraph will not start the system with batman-adv running. The user must manually assign an IP address to the TUN/TAP by running: \\

\code{sudo ifconfig tun0 192.168.200.XXX} \\

Where the X's can be replaced with an IP Address of your choice. Make sure to keep the IP's unique on each node. Then, batman-adv must be started on each node. To do this, a helpful script called ``raiseBatSignal.sh'' can be run to automatically configure batman-adv. This is described below. 

\subsection{Raising the Bat Signal}

``raiseBatSignal.sh'' can be found in: \\ 

\code{ARCAM-Net-Public/WebInterface/static/shell/} \\

It can be run by typing: \\

\code{sudo sh raiseBatSignal.sh} \\

Alternatively, you can run the commands individually from the command line. This shell script ensures the batman-adv module is loaded into the kernel then configures tun0 to be the interface used by batman. If this does not match your setup, adjust tun0 to be whatever interface the SDR is named. The contents of the shell script are below:

\lstinputlisting[language=bash, breaklines=true, caption={Shell Script to Start Batman-adv}]{/home/john/ARCAM/WebInterface/static/shell/raiseBatSignal.sh}

\subsection{TMux Script}

TMux is a terminal multiplexer that makes it easy to work with numerous ssh instances across multiple machines. A script is provided that sets up the entire network by running just a single command. TMux needs a bit of adjustment to work with different setups. Luckily, the only changes that need to be made are to open the file and edit the IP addresses, usernames, and passwords of the systems you want to control. This script assumes each node will have the same username and password. If network policies restrict you from doing this, then the script will need heavier modification. This script cant be found in: \\

\code{ARCAM\-Net\-Public/Tools/tmux\_start.sh} \\

The contents of the file are shown below. To run the command use: \\

\code{sudo bash tmux\_start.sh} \\

Please note that this script must be run with bash in order to work. 

\lstinputlisting[language=bash,breaklines=true,caption={Bash Script to start the entire network}]{/home/john/ARCAM/Tools/tmux_start.sh}




\section{Full Web Server}



%\include{Appendices/AppendixB}
\chapter{About the Author}
\label{aboutAuthor}
\includepdf[pages=-]{JMcCormack_CV_2-16-2015.pdf}




\end{document}  
