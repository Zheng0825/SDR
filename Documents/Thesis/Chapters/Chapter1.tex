% Chapter 1

\chapter{Introduction} % Main chapter title

\label{Chapter1} % For referencing the chapter elsewhere, use \ref{Chapter1} 

The Advanced Radio Communication Ad-Hoc Mesh Network, or ARCAM-Net, is a platform formed through the union of several other projects and concepts \cite{selfpaper}. The main concepts embodied within ARCAM-Net are mesh networks and software defined radio networks. The current iteration of ARCAM-Net is developed using GNU Radio and Batman-adv. As work progresses on the platform, the software defined radio network can be transitioned to a cognitive radio network by leveraging the capabilities found within GNU Radio. 

%----------------------------------------------------------------------------------------

% Define some commands to keep the formatting separated from the content 
\newcommand{\keyword}[1]{\textbf{#1}}
\newcommand{\tabhead}[1]{\textbf{#1}}
\newcommand{\code}[1]{\texttt{#1}}
\newcommand{\file}[1]{\texttt{\bfseries#1}}
\newcommand{\option}[1]{\texttt{\itshape#1}}

%----------------------------------------------------------------------------------------

\section{Mesh Networks}

In a traditional Wide Area Network (WAN), a user is able to connect to the internet through an Internet Service Provide (ISP). The user will almost always pay the ISP in exchange for the ability to connect to the rest of the internet. This is considered a centralized way to connect to the internet, where the users all connect through a few central points in the ISP's infrastructure. An alternative to this type of networking is multi-hop, ad-hoc, mesh networking. Mesh networks are decentralized networks \cite{4796928}. 

In an ad-hoc network, each radio is able to communicate directly to any other radio within its transmission range. There is no need to connect to a central router \cite{4796928}. Ad-hoc networks are defined at the physical layer (PHY), or layer 1 in the Open Systems Interconnect (OSI) model. Mesh routing takes place as part of layer 2 or 3 of the OSI model depending on the routing protocol chosen. A mesh network builds upon an adhoc network by allowing radios to retransmit any packets they receive. This allows two radios to communicate over a larger distance by leveraging other radios located in between the sender and receiver\cite{0033}. 

In simple mesh networking protocols, any packet sent may flood through the network to every other radio. However, with more advanced protocols an algorithm is used to ensure that a packet follows a direct path from sender and receiver and only uses a hop if necessary. The distributed nature of a mesh network creates many unique features. The decentralized nature of a mesh network prevents issues related to single points of failure. If a node goes down, the network can reconfigure and find a new path to the target \cite{0033}. 

%----------------------------------------------------------------------------------------

\section{Software Defined Radio Networks}

Software Defined Radios (SDRs) are radio communication systems that utilize software to process radio frequency information in place of traditional hardware \cite{761033}. A radio frequency front end is able to capture and transmit signals, while the actual processing of the signal is taken care of by a digital system like general purpose processor on a traditional computer. This allows for a single piece of hardware to replace the need for multiple types of radios \cite{393001}.

A typical cell phone can have a bluetooth, wifi, gps, and cellular radio all in a very small package. In the future, these systems could be replaced by a single SDR \cite{393001}. SDRs are capable of using both digital and analog transmission protocols. They can use general purpose processors, digital signal processors \cite{393001}, or FPGAs \cite{5747366} to process the RF information. Analog to Digital Converters (ADCs) are used to receive data from the antenna while Digital to Analog Converters (DACs) are used to transmit the processed signals . 

As the name suggests, a Software Defined Radio Network (SDRN) is a network made up of SDRs. The networks can operate on a nearly infinite combination of center frequencies, amplitudes, bandwidths, and protocols \cite{7039225}. The flexibility of an SDRN is limited by the physical hardware capabilities of the SDR, the computation speed of the processing unit, and regulations from governing bodies like the Federal Communications Commission (FCC). Still, the flexibility of reconfigurable radios leads to opportunities for advancing communications infrastructure beyond traditional protocols. 

\section{Batman-adv}

The Better Approach to Mobile Ad-Hoc Networking Advanced (Batman-adv) protocol is a layer 2 mesh routing protocol developed by the Open Mesh foundation \cite{0032}. The protocol uses a Transmission Quality (TQ) metric to find a tradeoff between a low hop count and stable links. Every node on the network broadcasts an originator message to its neighbors. Each neighboring node listens for these broadcasted messages and uses the number received to create the TQ metric. Each node also broadcasts the names of its neighbors. This allows for the creation of a multihop network where packets can be forwarded through other routers until they reach their destination \cite{6115569}. 

\section{GNU Radio}

GNU Radio is an open source tool chain for digital signal processing \cite{7043470} . The GNU Radio Companion is a graphical programming tool for developing DSP systems. GNU Radio can be used for a variety of topics, but it has many tools for working with software defined radios \cite{7430125}. GNU Radio is free to use, and all of the source code is available for download under the GNU Public License \cite{6081528}. 
 
%----------------------------------------------------------------------------------------

\section{Trends Towards Cognitive Radio Environments}

Cognitive Radio Networks (CRNs) are systems of SDRs that are capable of utilizing artificial intelligence and machine learning to dynamically alter transmission patterns in real time \cite{Akyildiz2007921}. These decisions can be made by a central server that oversees all nodes on the network, but modern systems strive to make each node capable of independent decisions. 

Cognitive Radio Ad-hoc Networks (CRAHNs) combine software defined radios to form mesh networks \cite{Akyildiz2009810}. Cognitive features of the radios can allow them to change transmission parameters in accordance with link quality to ensure packets are routed properly in the mesh network. The CRs could make small changes, like increasing or decreasing their gains, in order to continue to transmit to moving nodes. They could also make larger changes, like switching entire protocols, to adapt to the needs of the network in near real time. 

Beyond ensuring good throughput in a network, CRNs also solve a major issue facing the wireless world. Frequency spectrum is a finite resource, and the available bandwidth is being quickly used up. CRNs can utilize frequency hopping to share frequency resources with traditional wireless communication systems  \cite{6892537}. A CRN can begin its operation on a specified band. If a traditional transmitter, usually called the primary user (PU), begins to operate on that frequency then the CRN will ``hop'' by switching to a different frequency and continuing operation. 

%----------------------------------------------------------------------------------------

\section{ARCAM-Net}

In order to begin work on SDRNs and CRAHNs, a testbed and research platform needed to be established. This thesis serves to describe the Advanced Radio Communication Ad-hoc Mesh Network, or ARCAM-Net, test bed and platform. The goal of this project is to create a open source, low cost, research platform that can serve as the basis for future work at Florida Polytechnic University and other organizations due to the open source nature of the project. By combining well established open source tools, ARCAM-Net could become a great tool for other research groups to begin working with SDRs. All software used by ARCAM-Net is freely available. The only costs come from the hardware. 

ARCAM-Net's architecture will be thoroughly discussed in this document, but the two major components are GNU Radio and Batman-adv. GNU Radio is an open source toolchain for creating digital signal processing tools to be used with SDRs. Batman-adv is a layer 2 mesh networking protocol. Both projects are community driven, but exist as separate entities. I hope that with ARCAM-Net the two projects can begin collaborating to help create next generation wireless network resources. 

ARCAM-Net is designed to be a platform. Therefore all of the code will be released on github. I encourage interested users to fork the repository and begin experimenting on their own. 

%----------------------------------------------------------------------------------------
