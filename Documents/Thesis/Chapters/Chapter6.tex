% Chapter 6

\chapter{Conclusion} % Main chapter title

\label{Chapter6} % For referencing the chapter elsewhere, use \ref{Chapter1} 

%----------------------------------------------------------------------------------------

% Define some commands to keep the formatting separated from the content 
%\newcommand{\keyword}[1]{\textbf{#1}}
%\newcommand{\tabhead}[1]{\textbf{#1}}
%\newcommand{\code}[1]{\texttt{#1}}
%\newcommand{\file}[1]{\texttt{\bfseries#1}}
%\newcommand{\option}[1]{\texttt{\itshape#1}}

%----------------------------------------------------------------------------------------

\section{Summary}

%----------------------------------------------------------------------------------------

\section{Conclusion}

This work represents a first step in a longer term project to create a fully functioning SDRN. The work demonstrates the potential for using GNU Radio in conjunction with batman-adv and other Open-Mesh solutions. As the work continues forward, we hope the testbed can serve as a collaboration point between GNU Radio and Open-Mesh. Both represent next generation, open source, wireless solutions and could likely benefit from collaboration between the two projects. Our work can be used as an excellent starting point for anyone looking to get an SDRN up and running quickly to begin prototyping other sections of the tool chain. We plan on releasing the code as well as a handful of tools to help other researchers get started. We hope that this will serve as the first step in creating an equivalent Cognitive Radio Network platform and testbed. 

%----------------------------------------------------------------------------------------

\section{Recommendation}

%----------------------------------------------------------------------------------------

\section{Future Work}

%----------------------------------------------------------------------------------------
