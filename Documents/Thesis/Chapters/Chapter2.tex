% Chapter 2

\chapter{Literature Review} % Main chapter title

\label{Chapter2} % For referencing the chapter elsewhere, use \ref{Chapter1} 

The following section provides an overview of the current work being done in this field. The first section provides an overview about Software Defined Radios. Next, tools for working with SDRs are presented and discussed. Then, various networking concepts are described with a focus on mesh networks. Then, current work on software defined radio testbeds are examined and compared to ARCAM-Net. Finally, the state of the practice and the state of the art are examined. 
%----------------------------------------------------------------------------------------

% Define some commands to keep the formatting separated from the content 
%\newcommand{\keyword}[1]{\textbf{#1}}
%\newcommand{\tabhead}[1]{\textbf{#1}}
%\newcommand{\code}[1]{\texttt{#1}}
%\newcommand{\file}[1]{\texttt{\bfseries#1}}
%\newcommand{\option}[1]{\texttt{\itshape#1}}

%---------------------------------------------------------------------------------------
\section{SDR Fundamentals}

A software defined radio, or SDR, is a radio system which performs signal processing in software instead of relying on dedicated hardware components to process signals. In a traditional hardware radio, each component is designed and built to handle a single task. If changes are needed, the hardware components must be replaced. Software defined radios are flexible and can be changed by altering code instead of components. A single SDR can be configured to act as numerous hardware radios. \cite{7043470} \cite{0018} 

An SDR itself is the hardware component which then offloads the data to a processing unit such as an FPGA, microprocessor, or general purpose computer \cite{7043470} \cite{0018}. There are many different SDRs available on the market but they share many common components. Every SDR works in slightly different ways, but they share many common components. 

An ``Ideal'' SDR only has a few stages. The first is the RF amplification stage. These stages increase the gain of the signal as it enters and leaves the radio. A Low Noise Amplifier (LNA) is used on the receiver and a Power Amplifier (PA) is used on the output. The PA dramatically extends the range of the SDR and the LNA increases the signal to noise ratio of the receiver \cite{761033}. 

In the ``digital conversion'' stage, an analog to digital (ADC) and digital to analog (DAC) converter are used to change from the analog to digital domain \cite{761033}\cite{0020}. The ADC receives the analog signal and converts it to a digital code. The DAC takes digital material and converts it to an analog signal to be transmitted. 

The final stage is the ``processing'' stage. In this stage the digital signal is sent to and from the software processor. This can come in many different forms including microcontrollers, FPGAs, and general purpose computers \cite{393001}\cite{0020}.  

In an ideal SDR, the DAC and ADC can receive and transmit any needed signal. In practice, a real SDR requires a ``frequency conversion'' stage \cite{393001} \cite{0020}. This is an analog preprocessing stage that allows the useable frequency range of an SDR to be increased. Without this stage, the SDR's useable frequency range would be limited by the ADC and DAC to a much smaller range. 

SDR has numerous benefits compared to Traditional Radio. A single SDR can replace a multitude of radios, saving on space and cost. The operating frequency of an SDR is also flexible. This indicates that a radio can be moved from country to country and only the software and antenna needs to be changed to operate on new frequencies \cite{393001} \cite{0019}. As new standards, protocols, and technologies are developed, an SDR can download new software to adjust its operating parameters. Compare this to existing Wi-fi routers, where a person needs to purchase a whole new router as new standards are created \cite{761033} \cite{0019}. 

%----------------------------------------------------------------------------------------

\section{GNU Radio and MATLAB}

There are several different tools available for creating SDR software. A few different versions are presented below. 

\subsection{GNU Radio}

GNU Radio is a free, open source, digital signal processing toolset aimed at allowing for quick and easy development of software defined radio programs. GNU Radio is licensed under the GNU General Public License (GPL) version 3 \cite{0019}. GNU Radio itself is made up of a combination of python and C++ modules. It relies heavily on the Vector-Optimized Library of Kernels (VOLK) \cite{0021} and the Boost C++ libraries \cite{0022}. 

GNU Radio programs can be developed in python or using the GNU Radio Companion \cite{0023}. GNU Radio Companion (GRC) is a graphical tool for creating signal flow graphs and can also be used to generate python code from the flow graphs. GRC is useful for creating working SDR software with minimal programming experience. It also allows for the development of parallel processing without having to understand the details of keeping everything thread safe \cite{0023}.

Custom GRC blocks can be written in C++. Flowgraphs developed in GRC can also be converted into custom blocks by first converting them to hierarchical blocks. GRC can also create interactive GUIs by leveraging Qt or Wx \cite{0023}. 

GNU Radio and GNU Radio Companion can be extended by using out of tree (OOT) modules \cite{0024}. These are a set of modules that are not part of the core feature set of GNU Radio, but have been identified as important by the community. The official OOT repository is hosted by the Comprehensive GNU Radio Archive Network (CGRAN) \cite{0025}. GNU Radio also provides the PyBOMBS tool to create an ``app store'' style interface for downloading GNU Radio tools \cite{0024}. 

GNU Radio has a large community of active users. A mailing list and internet relay chat (IRC) channel are available for asking questions. Source code is updated often and anyone can contribute to the process \cite{0003}. Though hardware is not specifically part of GNU Radio, numerous vendors such as Ettus Research and Nuand offer support libraries for working with GNU Radio \cite{0026}. 

\subsection{MATLAB}

MATLAB and Simulink are another set of tools used for working with software defined radios. MATLAB and Simulink are commercial options developed by Mathworks. They are not free or open soure. The two pieces of software provide similar functionality to GNU Radio, but give you the ability to leverage Mathwork's support staff and knowledge base \cite{0027}. For this project, MATLAB was evaluated but ultimately not used. The decision was made to use as many free and open-source software (FOSS) components as possible. 

\subsection{RFNoc}

Ettus Research recently released the RF Network on Chip (RFNoC) extension to GNU Radio. RFNoc allows for the creation of FPGA code blocks. Unlike normal GRC blocks, these FPGA blocks compile into code that can then be run on the FPGA located on an SDR. This can dramatically increase the speed of the block. Users can create additional RFNoC blocks in VHDL and Verilog. Unfortunately, RFNoC is only compatible with X300 and E300 series SDRs, so the software could not be used for ARCAM-Net \cite{0028}. 

%----------------------------------------------------------------------------------------

\section{Network Concepts}

A Network consists of two or more computers that are connected together in a way that allows them to share resources, transfer files, and communicate electronically. The connection between them could be physical or wireless. Traditional computer networks include local area networks (LAN) and wireless local area networks (WLAN) \cite{529072} \cite{0029}. 

\subsection{Mesh Network}

Mesh networks, sometimes referred to as Mobile Ad-hoc NETworks(MANETs), are a type of network in which routers connect directly to nearby routers and connect indirectly to far away routers by ``hopping'' through the nearby routers until the packets reach their destination. Mesh networks can be created independent of traditional internet service providers (ISPs) as anyone with a router is able to join and create their own node on the network \cite{6908725}. 

Instead of using traditional and expensive network infrastructure, mesh networks rely on sheer number of nodes to create reliable and distributed networks. If a single node fails, the network can reconfigure and use other nodes to route packets to their destination. Expanding the network is as simple as adding more nodes to the edges of the network \cite{6908725}. 

\subsection{Mesh Network Testbeds}

The CONFINE platform uses batman-adv as the routing protocol for their mesh network testbed. However, their testbed does not utilize GNU Radio or SDRs. \cite{0001} batman-adv was also a key component of WiBed, a project to create a Commercial off-the-shelf (COTS) mesh test bed using low cost wireless routers \cite{6686492} \cite{6962154}. Though they are not specific to SDR platforms, these still show the usefulness of batman-adv as a component of a testbed. 

\subsection{GNU Radio Mesh Networks}

There has been work done in the past on establishing mesh networks using GNU Radio, however most of the research has had a different focus than ours. In \cite{4509617} and \cite{5062250} the authors use GNU Radio as a way to verify the successful use of algorithms for mesh networking. However, they do not use SDRs, instead choosing to use GNU Radio for simulation. 

The researchers in \cite{7141228} created a simple multihop test bed using three USRP radios to relay data from one computer to another. A fourth USRP acts as a primary user and attempts to block the signal. However, their work focuses on using reinforcement learning to allow for frequency hopping instead of focusing on a mesh routing protocol that would be able to scale to more radios. 

Much of the existing work done using USRPs and GNU Radio for SDR Mesh Networks revolves around implementing different parts of the Open Systems Interconnection (OSI) protocol stack from the ground up. In some papers the authors focus on the physical or mac layer \cite{5508221}. There has also been work in developing new higher layer protocols for cognitive radio mesh networks such as work done to replace Transmission Control Protocol (TCP) with a more robust protocol \cite{6686523}. These systems will usually react to frequency changes but some also change their topology based on power use \cite{6983150}. Power based topology changes will often be performed in order to lower the power cost associated with transmitting a packet. A shorter range will allow for the packet to be sent at a lower transmission amplitude which can save resources in a mobile or constrained environment \cite{6983150}.

\subsection{What is a network protocol}

A protocol is essentially a set of rules that govern the communication between two devices. A network protocol is used to define the way multiple computers can electronically send messages to one another. By standardizing protocols, computers and devices from various manufacturers can still communicate effectively \cite{6840086} \cite{0029}. 

\subsubsection{OSI Model}

The Open Systems Interconnection (OSI) Model is an abstraction that breaks network communication into seven layers \cite{6840086} \cite{0030}. Each layer has a standardized way of communicating with the layers above and below it. This indicates that at any given layer, a protocol can be exchanged for another without making any change to the other layers\cite{6840086}. 

The first layer is the physical layer. This layers deals with the transmission and reception of unstructured raw bit streams over physical mediums. Common Layer 1 protocols include ethernet and 802.11 \cite{6014631} \cite{0030}. Layer 2 is the data link layer. This layer passes structured data to Layer 1, and passes recognized packets on to layer 3 \cite{6840086}. A common layer 2 protocol is Medium Access Control (MAC) \cite{6014631} \cite{0030}. 

Layer 3, or the network layer, deals with routing information through the network. A common layer 3 protocol is Internet Protocol version 4 (IPv4) \cite{6014631} \cite{0031}. Layer 4 is the transport layer. Layer 4 can be used to ensure that messages are delivered error free and in sequence. Common layer 4 protocols are Universal Datagram Protocol (UDP) and Transportation Control Protocol (TCP). The next three layers are Session, Presentation, and Application layers \cite{6840086} \cite{0030}. ARCAM-Net does not make any changes to these layers so they are less pertinent to the discussion. 

Layer 5, or session layer, allows for the establishment of sessions between processes running on different stations. Layer 6, or presentation layer formats the data to be presented to the application layer. The final layer, application layer, passes the data off to the user \cite{6840086}. 

\subsection{Common Mesh Network Protocols}

There are a variety of network protocols focused on the creation of Mesh Networks. Unicast routing protocols are one family of mesh network protocols. One form of these protocols is the practice unicast routing protocol \cite{4796928}. In these networks a ``table'' is created by continuously evaluating routes to all reachable nodes. The protocol will try to keep the table as up to date and accurate as possible \cite{4796928}. When a network topology changes, the changes must be propagated throughout the network. OLSR borrows from this concept \cite{0033} \cite{5375690}. 

Another unicast method is called reactive unicast routing.These protocols are sometimes called ``on-demand'' unicast routing protocols \cite{4796928}. This type of routing finds routes only when desired by a source node. Only then does it initiate a routing discovery process. This can sometimes cause disconnect in mobile networks, but reduces the overhead associated with a proactive routing protocol \cite{0033}.

Multicast routing protocols improve on unicast methods by making networks more robust to mobile nodes that can change their location in a network. There are three main categories of multicast networks \cite{4796928}. The first is the simplest, in which packets ``flood'' the network. Every node that receives a network passes the data on to every node it is connected to. This causes an exponential growth in the number of packets on a network for every packet sent. The second is a proactive approach which pre-computes paths to all nodes and then routinely updates and distributes the routing table \cite{4796928}.

In the final method, paths are created on demand. First, a source searches the mesh for the end destination. Then it computes the shortest path and sends on this path \cite{4796928}. Batman-adv is a hybrid model that borrows ideas from both multicast and unicast protocols. It is described as a ``biologically inspired'' network protocol \cite{0033}.  

\subsubsection{Batman-adv}

Batman-adv is a layer 2 mesh routing protocol developed by Open-Mesh. By running on layer 2, batman-adv becomes layer 3 agnostic. This allows most applications to utilize batman-adv without making any changes to the code. Batman-adv has been integrated into the linux kernel and can be found packaged with many popular distributions of linux \cite{5375690}. 

Batman-adv is a proactive routing protocol aimed at using minimal resources in order to work with embedded hardware. 
Batman-adv uses the principle that better links will provide fast, reliable connections. Every node periodically sends out a broadcast message to inform its neighbors of its existence. The neighbors then relay this information along to the rest of the network. The link quality metric is established by detecting where these broadcast packets fail \cite{5375690}. 


\subsubsection{OLSR}

The Optimised Link State Routing (OLSR) protocol uses a ``link-state'' algorithm to proactively find efficent paths between nodes. The network works by using Multi-Point Relays (MPRs) that increase network throughput by creating efficient network schemes. OLSR does this by selecting a subset of neighbouring nodes to relay data instead of every node acting as a relay. This minimies the number of packets required to establish a routing table. Network information is shared between MPRs. Every MPR has a complete routing \cite{5375690}. Batman-adv was created by Open Mesh to fix problems they saw with OLSR, namely creating an algorithm better suited to embedded system hardware \cite{0032}. 

\subsubsection{802.11s}

802.11s is an extension of the 802.11 standard to include capabilities for multihop mesh networking \cite{6178212} \cite{5483777}. 802.11s is a new standard that has been rewritten over the past few years. As such, it is hard to find a good comparison between 802.11s and batman-adv as many comparisons are based on older, draft versions of 802.11s. 

802.11s combines AODV (Adhoc On-Demand Distance Vectoring) with a proactive protocol to create a Hybrid Wireless Mesh Protocol (HWMP) \cite{6379142}. AODV keeps packet overhead small by only finding routes when a route is needed. However, the tradeoff is higher computational complexity. AODV also takes more time to build a routing table than other protocols like OLSR and Batman-adv. To improve upon the performance of AODV, 802.11s uses the proactive route generation to create a predefined path to all mesh nodes. If the node moves, or can no longer be found, then 802.11s will use AODV to find a new path \cite{5483777}. 

The open80211s project is an open source implementation of 802.11s. Currently, it only supports the AODV protocol so benchmarks cannot fully characterize the effectiveness of this standard \cite{5483777}. Testing has revealed that 802.11s has better node recovery time and lower packet loss than batman-adv. However batman-adv has much higher throughput and lower power requirements \cite{5483777} \cite{6379142}. 802.11s is also only compatible with 802.11 hardware. In order to implement it on an SDR, the radio would need to be configured as an 802.11 radio. Batman-adv is agnostic to layer 1 and can be configured to run on any standard such as ethernet \cite{5483777}. 802.11s is also designed to work with around 32 nodes \cite{ieee802tut}. 

%----------------------------------------------------------------------------------------

\section{Software Defined Radio Testbeds}

There are several well-known Software Defined Radio testbeds in use at different Universities. One major platform is the WARP platform from Rice University. This platform is made up of many custom components including the radio hardware itself \cite{7071706}. This makes the platform very expensive and limits its adaptability for use in other research facilities. 

Another platform is the Hydra platform developed at UT Austin. This platform uses GNU Radio to define PHY Layer parameters and the Click Modular Router to implement Layer 2 protocols \cite{4212821}. The Hydra platform also uses USRP radios as the hardware frontend. However, Click is an older software which, according to their website, has not had an updated release since 2011 \cite{0009}. GNU Radio has since added the Polymorphic Tree (PMT) and Message types to allow for more Layer 2 development to be done right inside GNU Radio \cite{0010}. 

The ADROIT project was another platform developed in conjunction with DARPA. This project relied heavily on Click and GNU Radio for much of its functionality. \cite{4286321}  Similarly, the University of California, Irvine and Boeing Corporation developed a testbed based on USRP Radios and GNU Radio, but they implement custom MAC layers \cite{4753441}. 

A platform similar to ARCAM-Net is presented in \cite{0002}. However, this platform uses the Optimized Link State Routing Protocol (OLSR) which has been shown to perform poorly when compared to batman-adv. OLSR is also not truly decentralized as only certain nodes relay network information \cite{5375690} Rutgers University also has a platform called the ORBIT test bed. This platform is a massive mix of over 100 nodes. However, not all nodes are equipped with SDRs. Also, ORBIT is meant to be used by other researchers, not replicated. Therefore, instructions for building the platfrom yourself are not provided. They also do not provide a working software backend, the user is expected to test their own waveforms and networking protocols \cite{1386189}. 
  
%----------------------------------------------------------------------------------------

\section{State of the Practice}

\subsection{RTL-SDR}

	The RTL-SDR is an SDR tool discovered by the DIY and hacking community. Its original purpose is to be used as a digital TV tuner. However, it was discovered that this system could also be used to general SDR purposes. There is now a large community dedicated to using this tiny SDR to receive various different signals. Prior to the creation of the open source drivers for the RTL-SDR, the most popular devices for SDR came from USRP. The USRP devices are fantastic products, but cost at least \$1,000 and can cost quite a bit more with additional features. The RTL-SDR is based on the Realtek RTL2832U chip. This device can often be purchased for between \$20 and \$30 \cite{6526525}. The range for the RTL-SDR is typically 64 MHz	to 1700 MHz, however this varies depending on which tuner the manufacturer paired with it. The authors in \cite{6526525} paired the RTL-SDR with a mixer in order to lower the range all the way to DC. For this, they used the NE6062AN chip.  
	
	Starting with Release 2013b, MATLAB/Simulink now have a support package that targets RTL-SDR devices. In Simulink, the package contains a single block called ``RTL-SDR Receiver.'' This block allows the user to tune the center frequency, change the tuner gain, set the sampling rate, and alter the frequency correction factor. The block can then output the complex envelope (IQ) of the received signal in both floating point and integer formats\cite{6893337}.  Due to the open nature and low cost of the RTL-SDR, the authors in \cite{6821718} propose using this as a tool set for teaching DSP and Communications principles to students. 
	
	When the cost of the system is far less than that of a
	textbook, it is easy to understand why this could become a valuable learning tool for many students. UC Berkley has already begun to use the RTL-SDR as one of the project assignments in their digital signal processing course. 	There have been efforts to use the RTL-SDR with the popular Raspberry Pi computing platform. However, at least with the B+ model there is not enough power available to process the signal. Instead, it has to be used as a TCP server that is then able to forward the data onto a more powerful computer\cite{6938691}. Currently, it does not appear as though anyone has tested this with the Raspberry Pi 2 microcomputer. Other work has been done to estimate the cost savings of using a USRP in conjunction with several RTL-SDRs to replace existing DSP lab infrastructure \cite{6726630}. 
	
	The most unfortunate downside to using the RTL-SDR is that it is unable to transmit. However, researchers in \cite{6922233} proposed a system in which a USRP SDR is used as a master device that broadcasts out to a series of slave nodes that can only listen for information. In their system they used a tool called GStreamer to pass video data into GNURadio. This information was then broadcast to multiple computers in a room running an RTL-SDR with GNU Radio. The nodes were all able to view the video stream in near real time.  

\subsection{USRP}

ARCAM-Net uses the Universal Software Radio Peripheral (USRP) created by Ettus Research, a division of National Instruments\cite{0006}. Ettus also released the Universal Hardware Drivers (UHD) \cite{0007} which allow for the use of the USRP with GNU Radio. USRP radios are supported by both Mathworks \cite{0027} and GNU Radio \cite{0026}. Ettus research provides the open source USRP Hardware Drivers (UHD) to allow the devices to interface into various tools \cite{0007}.

The USRP line of radios are able to transmit and receive data. The B200 and B210 operate over USB 3.0 \cite{0034}. The E310 radios are standalone systems with their own onboard processor built into the FPGA \cite{0035}. The newer E310's also come with an onboard battery, allowing them to operate in a mobile fashion \cite{0036}. The USRP line of radios is used heavily in academia. The radios form the basis of numerous testbeds  \cite{4753441} \cite{4212821}. They are also used to validate data from simulations \cite{5508221}. 

\section{Industry State of the Practice}

General Dynamics produces the S-Band software defined radio. This radio is used in the Space Communications and Navigation Testbed on the International Space Station. This SDR provides experimenters an opportunity to develop and demonstrate experimental waveforms in space \cite{6497193}. 

SDRN and CRN devices are ideal for emergency response and military operations. Various countries operate using different standards for radio transmission. SDRs and CRs allow for better coordination, as the radios can be configured to operate using a form not commonly used by a given organization. In emergency response, the parameters of the radio can also be changed to provide additional bandwidth and spectrum for the duration of the operation \cite{5639025}. 

Harris also develops commercially available SDRs. Harris began work in 1998 using RISC based processors for their Software Defined Payloads (SDP) \cite{5747366}. Since then, they have switched to FPGAs to offer greater flexibility. Harris's SDPs are deployed in numerous Space Telecommunications Radio Systems (STRSes). By using an SDR with an FPGA, Harris is able to remotely update and improve the capability of satellite communication devices \cite{5747366}. 

\subsection{Software Defined Networks}

Software Defined Networking (SDN) is a new paradigm that moves a network's control logic to a central location and away from the underlying hardware \cite{6994333}. This allows for greater flexibility in how the network is designed. The control platform can be configured to move data more predictably and with finer control over the data. In traditional IP networks, the control and data planes are tightly coupled \cite{6994333}. Each router or switch will have both capabilities on board will need to be configured individually \cite{7179430}. This leads to fast routing, but only if each router is configured properly. A single router performing in an undesirable way may be near impossible to detect as the administrators will need to scan each router to find the cause. 

Software Defined Networking separates the control and data plane from one another \cite{6994333}. The hardware devices then become simple packet forwarding elements. Forwarding decisions become flow based instead of destination based. Instead of routing tables, flow tables define how groups of packets should be treated when they reach a given piece of hardware \cite{7179430}. The control logic is removed from the hardware devices and instead an SDN controller is used \cite{6994333}. This controller is a software platform that runs on a server and provides the essential resources and abstractions needed to program the forwarding devices. Software applications running on top of this network controller allow for the administrators to interact with the data plane \cite{7452335}. 

\subsection{SDRN vs SDN}

SDRNs and SDNs have a similar purpose, but take very different approaches. They both seek to increase flexibility in communication, but at different layers in the OSI model. SDRNs focus on flexibility in layer 1 and 2 \cite{Akyildiz2007921}. SDN's focus on flexibility in layers 3 and 4 \cite{6994333}. Software Defined Networking is not a part of ARCAM-Net. However, as OSI layer abstraction is maintained, SDN platforms that are compatible with the hardware running the SDR could be used \cite{6933795}. 

%http://ieeexplore.ieee.org.flpoly-proxy.flvc.org/xpl/articleDetails.jsp?arnumber=5747366&newsearch=true&queryText=Harris%20Corporation%20SDR

%----------------------------------------------------------------------------------------

%----------------------------------------------------------------------------------------

\section{State of the Art}

%----------------------------------------------------------------------------------------

\subsection{Cognitive Radio Networks}

Cognitive Radio Networks represent a technological evolution from Software Defined Radio Networks \cite{6599059}. Cognitive Radio Networks (CRNs) are networks made up of SDRs that are capable of sensing their environment, making decisions, and changing transmission parameters \cite{Akyildiz2007921}.In a traditional Software Defined Radio Network the waveforms and transmission protocols are flexible. However, they are predefined by the software loaded onto the computer or radio. To change operating parameters, a different program is run or the user manually alters transmission patterns. In a Cognitive Radio Network, changes can be made intelligently through the use of machine learning and artificial intelligence. 

A major benefit of Cognitive Radio Networks is that they can use Dynamic Spectrum Allocation (DSA) \cite{6892537}. Current wireless transmission is performed on static, licensed bands. The spectrum must be purchased through the Federal Communications Commission (FCC) and the user can then only operate on that particular frequency. With Dynamic Spectrum Allocation, radios are able to find bands of frequency that are not currently being utilized and perform their transmissions there. Once a licensed user begins transmitting on that frequency, the radios can shift to a new band and continue operation. Dynamic Spectrum Allocation helps to alleviate the problems of spectrum sharing and drastically increases the usability of each frequency band \cite{6599059}. 

As with SDRs, Cognitive Radios are able to alter operating parameters such as transmission power, frequency, modulation type. Unlike SDRs, any of these parameters can be changed through the use of Artificial Intelligence and Machine Learning. The general cognitive functions of CRNs occur in what is called a ``cognitive cycle.'' \cite{5639025} This cycle is made up of four components: Sensing, Analysis, Reasoning, and Adaptation. In spectrum sensing and analysis, the radios are constantly scanning over a given frequency band, looking for openings. In the reasoning phase, the radio looks at the data for each frequency band and quantifies the ``ideal'' band to begin transmitting on. In the adaptation phase the radio switches to the new operating parameters \cite{5639025}. 

Each radio may be capable of performing these actions on their own, or they may leverage a spectrum broker. The spectrum broker is a centralized cognitive radio that may have wider band sensing capabilities \cite{5639025}. This central broker will then send commands to the other cognitive radios to dictate which frequency they should continue to transmit on. This decentralized approach is less ideal, as it limits the scalability of the network. However, it can often be less computationally complex to deploy \cite{5639025}. 

Cognitive Radio decision making can be made at nearly any level of the OSI model, having different effects on the overall network operation. Layer 1 decision making can be used to enable Dynamic Spectrum Sharing (DSS) \cite{5771952}. As wireless technology grows, more and more spectrum is being allocated. However, the world is quickly running out of useable spectrum. Cognitive Radios and DSS allow for cooperative sharing of available frequency \cite{5771952}. 

Primary Users (PUs), or those that have purchased a license to use the band, can operate on a given frequency \cite{4562561}. If there are any gaps in their transmission, then a secondary user (SU) can quickly transmit and receive on the frequency and then switch to a new frequency before the PU begins transmitting again \cite{4562561}. 

Layer 1 decision making can also be used to avoid congestion. If a Cognitive Radio detects there are a lot of users operating on a given frequency, then it can switch to a different channel to continue transmitting.\cite{Akyildiz2007921} Cognitive decision making is not limited to Layer 1. Layer 2 cognition could be used to reroute the paths used to transport packets. This could be done to improve throughput or reduce energy use \cite{6527405}. Layer 3 cognition could adjust signal gains if packets are being dropped too frequently \cite{6072038}. Layer 4 could detect when packets are going directly to a single node, and adjust the nodes frequency so they have their own independent channel of operation \cite{5062176}. In general, Cognitive Radios improve upon SDRs in that they can react to and predict changes in order to refine the operation of the network. 

\subsection{Cognitive Radio Ad-Hoc Networks}

Cognitive Radio Ad-Hoc Networks (CRAHNs) are essentially mesh networks made up of intelligent software defined radios \cite{Akyildiz2009810}. They are a specialized subset of Cognitive Radio Networks \cite{5639025}. CRAHNs seek to combine the flexibility of Cognitve Radios with the distributed network routing of mesh networks. CRAHNs allow for dynamic route topology due to the ability of ad-hoc networks to connect to any and all peers within their broadcast range. The features of the radio, such as frequency and amplitude, can be adjusted to force the topology of the user's choosing. This can also be done dynamically using artificial intelligence and decision making \cite{Akyildiz2009810}. 

Routing in multi-hop CRNs or CRAHNs has to deal with a number of challenges. The dynamic changing of the available spectrum can alter the topology of the network as different frequencies react to the environment in different ways. The routing metrics for CRAHNs will need to be able to adjust each time a hop is performed in order to rebuild the topology \cite{6599059}. The routing protocol will also have to be able to handle the transmission delay that will occur as the nodes are changing frequency \cite{6599059}. During the change, the network will be unreachable. The routing protocol will need to be able to begin transmission again just as the network comes back online or risk dropping packets. 

\subsection{Cognitive Networks}

Just as SDRNs can also be CRNs, SDNs can have cognitive features. These are simply referred to as Cognitive Networks \cite{7207253}. In these networks, the central control plane can reroute packets to distribute a load more evenly throughout a network \cite{6982928}. If one router is receiving much more data than the rest of the network, artificial intelligence can be used to begin to off load some of these packets to other routers. Many of the algorithms used to find empty spectrum in Cognitive Radio Networks can be applied to find more efficient routes within software defined networks \cite{7207253} \cite{6982928}. 



\subsection{Biologically Inspired Protocols}

These protocols are based on algorithms that are designed to mimic naturally occurring phenomenon. This leads to novel and efficient algorithms well suited for the distributed nature of a mesh network.

\subsubsection{Swarm Protocols}

Swarm Intelligence is a class of algorithms that are based around the social behaviors of insects. They are highly applicable to mesh networks as they are designed for distributed and decentralized problem solving. Adaptive Swarm-based Distributed Routing (Adaptive-SDR) is one type of swarm routing protocol that can be used for SDR applications. In this protocol, nodes are divided into colonies. Each node learns how to get to the other colonies, and how to get to nodes within their own colonies. When a packet traverses the network, each node forwards the data on the path to the correct colony first. Then nodes within the colony forward on to the correct node \cite{1005496}. This protocol is similar to the one used by Batman-adv \cite{0033}.


\subsubsection{Hive Protocols}

Hive protocols emulate the behavior of bees that forage and return to a central hive \cite{6863008}. In this network structure, the network is divided into several foraging zones. To find a path to a node, a bee (in the form of a packet) will seek out a destination node. Once the packet reaches the destination, another one will be sent back to the source on the same route. When it reaches the source this packet will then communicate the distance, direction, and quality of the path between the two nodes. If multiple bees are sent out, they will then be able to communicate the optimum route to take for future packets \cite{7373160}. 

\subsubsection{Ant Colony Optimization Protocols}

Ant Colony Optimization (ACO) Protocols are modeled off of ants cooperating to find food for their colony \cite{7030885}. In this protocol, a set of ``Ants'' are sent out in the form of discovery packets. The first set try to find the source node within the network. The other set try to find the destination node. As the ants walk, they deposit pheromones within the network to mark the nodes they have traversed. As more ants visit a given node, the quantity of pheromones at that location will begin to increase. After a while, the optimal path between a source and destination node can be determined by following the path with the largest amount of pheromones \cite{7030885}. 

\subsection{Decision Making and Biologically Inspired Decision Making}

At a very simple level, decision making can just be automatic gain correction (AGC). A cognitive radio with AGC may adjust its gain to its highest level until it begins to receive data. At the highest gain level, it may detect clipping and begin to lower its gain until an undistorted signal, with a good signal to noise ratio, is achieved. \cite{6497193}

Genetic Algorithms can be applied to MIMO cognitive radio systems. In these CR environments, a genetic algorithm is used to optimize bandwidth, band efficiency, transmission power, data rate, and bit error rate \cite{7124804}. The genetic algorithm works by using a fitness score to indicate the quality of the established link. Combinations of parameters that generate high fitness scores are allowed to ``breed'' into new generations of parameters. Eventually, an optimal set of parameters is discovered and the Cognitive Radio continues to transmit and receive at these settings \cite{7124804}.

Biologically-inspired algorithms are those that mimic nature in their execution. Biologically-inspired spectrum sharing is a major component of Cognitive Radio Networks. One such algorithm is called ``BIOlogically-Inspired Spectrum Sharing'' or the BIOSS algorithm \cite{4224259} \cite{5686503}. This algorithm uses a task allocation model based on insect colonies. The various parameters that an SDR can monitor are given response thresholds. If this threshold is passed then the radio will react and change to a new parameter. For example, if a large enough primary user began to use the spectrum the cognitive radio is transmitting on then it will switch to a new frequency \cite{4224259} \cite{5686503}. 

Another biologically inspired algorithm is used to form consensus-based spectrum sensing \cite{5464224}. This algorithm is based off of flocks of birds and group decision making of other types of animals. In this scenario, each node senses the spectrum around itself. When a given node begins to sense a primary user, it lets the nodes adjacent to it know \cite{5464224}. These nodes then propagate it further. Overtime, the entire network will ``steer'' towards a new frequency the same way a school of fish may collectively steer away from a predator. 


The Ant Colony Optimization (ACO) algorithm has also been applied to dynamic spectrum access in cognitive radio networks \cite{5600815} Each node explores possible bands that it can transmit in and then uses a simulated pheromone to indicate good locations to transmit on. These pheromones have a decay time to ensure that they are constantly updated. Once a band has a large number of nodes releasing pheromones on it, then the other nodes will converge on this frequency. This process can be repeated continuously to move away from primary nodes as they appear \cite{5600815}. 
