\section{Limitations and Future Work}

The results from our experiments show that the network is functioning as a multi hop SDRN. However, it is clear that more work needs to be done. In a deployed network, packet loss as high as is seen in this network would not be tolerated. Therefore, an important next step would be to examine how we can utilize machine learning and artificial intelligence to adjust parameters when these large packet losses are detected. It is likely that a change in frequency or amplitude could mitigate some of the packet loss. For example, if the loss is due to too many nodes operating on the same frequency, that set of nodes could change to a unique, unused frequency for the duration of the transmission and then switch back. This would be the beginning of the networks transition from a SDRN to a Cognitive Radio Network (CRN). 

Furthermore, it would be beneficial to either improve upon A.L.F.R.E.D. or reimplement certain features in a new way in order to handle the frequency changing. If we have each node wait for an acknolwedge from its immediate neighbors before changing frequency, that node could then change its operation knowing that the data will make it to the rest of the network. Batctl is already able to report the immediate next hop neighbors, so the program could use this information to only wait for acknolwedgements from neighbors instead of waiting for the entire network to be ready to change. A frequency change under these conditions would lead to a much more robust network change. In order for the current A.L.F.R.E.D. setup to function, a delay was needed to give the network time to respond. Therefore, an asynchronous acknowledge would likely speed up this transition as well. 

