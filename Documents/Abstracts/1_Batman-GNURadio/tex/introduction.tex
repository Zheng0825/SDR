\section{Introduction}

As the cost of SDRs continues to drop, the technology becomes much more accessible. Open source tools such as GNU Radio make developing for SDRs much easier \cite{0003}. GNU Radio provides a feature rich ecosystem that provides a wealth of signal processing blocks. Hardware is not a direct component of GNU Radio. However, numerous other developers and projects have integrated hardware functionality into GNU Radio, either natively or through additional out-of-tree modules \cite{0004} \cite{0005}.

GNU Radio Companion allows for GUI development of PHY and MAC layer protocols within the GNU Radio environment. The project itself is implemented in a combination of Python and C++ modules \cite{0003}. ARCAM-Net uses the Universal Software Radio Peripheral (USRP) created by Ettus Research, a division of National Instruments\cite{0006}. Ettus also released the Universal Hardware Drivers (UHD) \cite{0007} which allow for the use of the USRP with GNU Radio.   

Cognitive Radio Networks (CRNs) are networks made up of SDRs that are capable of sensing their environment, making decisions, and changing transmission parameters \cite{Akyildiz2007921}. Many Cognitive Radio scenarios are designed around the concept of ad-hoc or mesh network \cite{Akyildiz2009810}. In these networks, all of the associated radio components are able to talk to each other either directly or by ``hopping'' from one node to another until they reach their destination. The Open-Mesh project created the batman-adv protocol which allows for the formation of multihop mesh networks \cite{0008}. ARCAM-NET uses batman-adv as its layer 2 protocol. Batman-adv has a fairly large community and is integrated into the Linux kernel \cite{0008}. 

Our goal with this project was to create a low cost, open source platform that could serve as a basis for future projects in cognitive radio and mesh network research. Our hope is that by combining the GNU Radio and batman-adv projects, more collaborative research can be done with a standardized toolset. Our platform will allow for others to create application layer products that leverage software defined radio meshes to get started without having to worry about all the intricacies of these networks. We also hope to encourage both open source projects to work collaboratively in creating next generation wireless systems. 
