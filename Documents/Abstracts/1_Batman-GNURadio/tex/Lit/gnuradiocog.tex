\subsection{Using GNU Radio for Cognitive Radio Application}

Reaserachers have identified that the primary methods of creating a roboust CRAHN is by ensuring
that nodes optimize their use of physical space and allocated spectrum \cite{6846075}. A testbed is
needed in order to fully experiment with different algorithms to maximise these conditions. 
GNU Radio has been used by many different research groups to test various cognitive radio
standards. The researchers in \cite{7141228} created a simple multihop test bed using
three USRP radios to relay data from one computer to another. A forth USRP acts as a primary
user and attempts to block the signal. Their work focuses on using Reinforcement learning to allow
for the hopping and does not discuss the routing protocol used in much depth. Much of the 
existing work done using USRPs and GNU Radio for Cognitive MANETs revolves around implementing
different parts of the protocol from the ground up. In some papers the authors focus on
the physical or mac layer \cite{5508221}. There has also been work in developing new higher 
layer protocols for cognitive radio mesh networks such as work done to replace 
TCP with a more robust protocol \cite{6686523}. These systems will usually 
react to frequency changes but some also change their topology based on power use \cite{6983150}. 

There are several well known Cognitive Radio testbeds in use at different Universities. 
One major platform is the WARP platform from Rice University. This platform is made up of many
custom components including the radio hardware itself \cite{7071706}. Another platform is the
Hydra platform developed at UT Austin. This platform uses GNU Radio to define PHY Layer parameters
and the Click Modular Router to implement Layer 2 protocols.\cite{4212821} The platform that
most closely resembles ours is presented in \cite{0002}. However, this platform uses OLSR which 
operates on a layer above Batman-adv. Similarly, the University of California, Irvine and Boeing 
Corporation developed a testbed based off of USRP Radios and GNU Radio, but they implement
custom MAC layers \cite{4753441}. The ADROIT project was another platform developed in conjunction
with DARPA. This project relied heavily on Click and GNU Radio for much of its functionality.
\cite{4286321} Though not deployed in a cognitive radio
environment, the research in \cite{6115569} presents metrics on Batman-adv itself and
will be useful for seeing what decreases in performance are seen when using an SDR instead
of a traditional Wi-Fi Router. 
