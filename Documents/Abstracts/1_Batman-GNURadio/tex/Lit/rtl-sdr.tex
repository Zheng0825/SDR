\section{RTL-SDR}

	The RTL-SDR is a recent tool discovered by the DIY and hacking community. Its original purpose is to be used as a digital
	TV tuner. However, it was discovered that this system could also be used to general SDR purposes. There is now a large 
	community dedicated to using this tiny SDR to recieve various different signals. Prior to the creation of the open source
	drivers for the RTL-SDR, the most popular devices for SDR came from USRP. The USRP devices are fantastic products, but cost
	at least \$1,000 and can cost quite a bit more with additional features. The RTL-SDR is based on rhe Realtek RTL2832U chip.
	This device can often be purchased for between \$20 and \$30 \cite{6526525}. The range for the RTL-SDR is typically 64 MHz
	to 1700 MHz, however this varies depending on which tuner the manufacturer paired with it. The authors in \cite{6526525} 
	paired the RTL-SDR with a mixer in order to lower the range all the way to DC. For this, they used the NE6062AN chip.  
	
	Starting with Release 2013b, MATLAB/Simulink now have a support package that targets RTL-SDR devices. In Simulink, the
	package contains a single block called "RTL-SDR Receiver." This block allows the user to tune the center frequency, 
	change the tuner gain, set the sampling rate, and alter the frequency correction factor. The block can then output
	the complex envelope (IQ) of the recieved signal in both floating point and integer formats\cite{6893337}.  Due to 
	the open nature and low cost of the RTL-SDR, the authors in \cite{6821718} propose using this as a tool set for
	teaching DSP and Communications principles to students. 
	
	When the cost of the system is far less than that of a
	textbook, it is easy to see why this could become a valuable learning tool for many students. UC Berkley has 	
	already begun to use the RTL-SDR as one of the project assignments in their digital signal processing course. 
	There have been efforts to use the RTL-SDR with the popular Raspberry Pi computing platform. However, at least
	with the B+ model there is not enough power available to process the signal. Instead, it has to be used as a TCP
	server that is then able to forward the data on to a more powerful computer\cite{6938691}. Currently, it does not
	appear as though anyone hast tested this with the Raspberry Pi 2 microcomputer. Other work has been done to
	estimate the cost savings of using a USRP in conjunction with several RTL-SDRs to replace existing DSP lab infrastructure
	\cite{6726630}. 
	
	The most unfortunate downside to using the RTL-SDR is that it is unable to transmist. However, researchers in \cite{6922233}
	proposed a system in which a USRP SDR is used as a master device that broadcasts out to a series of slave nodes that can
	only listen for information. In their system they used a tool called GStreamer to pass video data into GNURadio. This
	information was then broadcast to multiple computers in a room running an RTL-SDR with GNURadio. The nodes were all able
	to view the video stream in near real time.  

	It is interesting to note that an IEEE search for the RTL-SDR turned up 6 results, but a search on Funcube Dongle returned
	0 results. These devices were in competition with one another for awhile, but the Funcubes prices continues to rise
	as the RTL-SDR's continues to drop. The Funcube has a larger range than the RTL-SDR but it seems the cost is still
	the key factor. 


