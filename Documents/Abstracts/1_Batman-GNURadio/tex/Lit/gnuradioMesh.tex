\subsection{GNU Radio and Mesh Networks}

In \cite{4509617} and \cite{5062250} the authors use GNU Radio as a way to verify the succesful use of algorithms for mesh networking. However, they do not specify that they are using SDR's and they simply use GNU Radio for simulation. GNU Radio has also been used with the USRP to create a device capable of communicating with both Bluetooth and WiFi devices. However, this does not create a mesh network or attempt to bridge communication between the two protocols. However, a significant amount of information about communicating with each type of network is presented \cite{4292880}. This also presents the concept of Police Nodes which monitor traffic in an attempt to block out improper use of the spectrum. 

Research has been done in using the GNURadio toolset along with the USRP to test Mesh Network routing protocols. One test used varied data transmission rates to exploit opportunities in physically close proximity Nodes \cite{5462112}. The GNU Radio toolset was also used to test using cognitive radio within a mesh network. USRPs were used as nodes trying to communicate on a
``shared'' frequency. A separate USRP was used to replicate a primary user, or one that had a license to operate in that spectrum. Whenever the primary user began to transmit in the spectrum, the other nodes would use reinforcement learning to move to an unoccupied channel automatically and continue transmitting \cite{7141228}. A similar test bed is also presented in \cite{5508221}.

In \cite{5984947} researchers at UCSB investigated using an SDR with GNURadio toimprove upon the needs of rural networks. The topic was found while search for mesh networks but seems to be mostly focused on non-mesh applications. They created the solution WhiteRate which allows for the changing of the PHY layer without changing any other components. 

In \cite{5462039} the authors utilize GNU Radio to implemenet a PHY layer that is able to broadcast and recieve on several channels simultaneously. The paper
tests using 2 USRP boards and also simulates a larger scale.  

The CONFINE platform uses Batman-adv as the routing protocol for their mesh network testbed. However,this testbed does not utilize GNU Radio or any cognitive radio tool sets. \cite{0001} Batman-adv was also a key component of WiBed, a project to create a COTS mesh test bed using low cost wireless
routers. \cite{6686492} \cite{6962154}
