%Software Defined Radio Networks (SDRNs)  use systems of Software Defined Radios (SDRs) to establish networks with flexible physical and link layers. GNU Radio is an open source software tool set for working with SDRs and can be used as a basis for creating SDRNs. Mesh networks are designed to allow for flexible and distributed network architectures that are self-forming and function without the need for centralized infrastructure. Batman-adv is a popular, open source, layer 2 mesh network protocol.

Software Defined Radio Networks (SDRNs) utilize systems of Software Defined Radios (SDRs) to establish networks with flexible physical and link layers, often in the form of self-forming, multi-hop networks called mesh networks. In this paper we present the Advanced Radio Communication Adhoc Mesh Network (ARCAM-NET) Platform. Our work establishes an SDRN platform by combining GNU Radio with batman-adv to create a fully open source software defined radio mesh network. The platform can work with USRP SDR devices to quickly prototype and potentially explore SDR and Cognitive Radio (CR) Protocols. 

Due to the flexibility of batman-adv and GNU Radio, programs acting above Layer 2 can utilize this network without any changes. In order to further increase the cognitive abilities of the platform, we explore using the A.L.F.R.E.D. tool chain within the batman-adv ecosystem to distribute information about frequency changes across the mesh network. This creates a method to globally change the frequency of the network in a completely decentralized way. 

