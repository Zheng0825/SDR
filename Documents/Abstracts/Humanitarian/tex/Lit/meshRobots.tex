\section{Mesh Networks and Robots}

Researchers have begun to examine the problems of using UAVs to communicate on a mesh network. Part of the problem lies in the fact that communicating
with a moving node can be difficult. One way to combat this is to use multiple nodes on the UAV that exploit the spatial and temporal diversity of the wireless
channel.They ran their tests on a 110" wingspan UAV with 4 mesh nodes on it. They used the 802.11 adhoc mode for their network. \cite{5700251} 
The same team presented a test of creating a network with a UAV using the Load, Carry, and Deliver paradigm. This UAV would "pick" up data from one node, fly to the
next node, and then drop it off. \cite{4225050} 

Work present in \cite{6564779} demonstrates a great overview of the current "state of the art" in regards to UAV
based networking. The author discusses the main routing formats and many of the challenges that go along with UAV network. This should be seen as a good starting
point for people first learning about this topic. In \cite{6842275} the author's discuss the concept of a cloud computing and software, platform, and infrastructure
as a service environment. They discuss abstracting out significant parts of the UAV system, such as navigation, so that they can be done in the cloud. This would
allow for most of the computation to be done in a single format off board, allowing for easier integrationg of heterogeneous swarms of robotics. 
