\documentclass{article}

\newcommand{\shellcmd}[1]{\\\indent\indent\texttt{\footnotesize\# #1}\\}
\usepackage{hyperref}
\usepackage{listings}
\usepackage[margin=0.5in]{geometry}

\title{Notes on SDR}
\date{2015-07-01}
\author{J.D. McCormack}

\begin{document}
	\pagenumbering{gobble}
	\maketitle
	\newpage
	\pagenumbering{arabic}
	
	\section{Introduction}

	The following document will contain information related to my efforts in software defined radio.
	From time to time, the document may include information on other topics as well. It will initially
	serve as a day to day journal of my activites. After a certain discovery phase, this document will
	be updated to create a manual or tutorial for future students to learn from. It is not recommended
	that the document be used for self study until a newer version is created. Many of the steps 
	that are outlined will be dead ends or poor practices while the disocvery process takes place. 

	\newpage

	\section{7-1-2015}

	Currently working on using the RTL SDR. 

	What I have learned so far: 
		\begin{itemize}
			\item This SDR is different than a FUNCUBE dongle
			\item It will work from around 500 Hz to 1700 Hz
			\item It can only RECIEVE 
		\end{itemize}
	The fact that this can only recieve is by far the largest setback.
	This will ultimately not be what we can use for cognitive radio. 

	However, the tutorials still seem more widely available than those
	found for the BladeRF which is bi-directional. Therefore I will
	continue for a bit. It may serve a purpose as part of the larger 
	sensor network at some point. We could still use it to relay information
	over a large distance. 

	There are "Cubelites" being sent out that are basically funcube dongle
	based satellites. These sound pretty cool. 

	I want to start taking notes in Latex so they will be more readable than
	a plain .txt file and still useable with version control. 

	I need to install texlive-full and texmaker. 

	Currently I am set back by needing to update ubuntu. I can't install
	anything else while the updates are installing. 
	
	\paragraph{Success}
		
		
	So far I have made decent progress. The RTL is working with GNU SDR. I had trouble
	at first when the kernel was loading a separate driver that bogs it down. 

	I used this command to fix that problem:
	\shellcmd{sudo rmmod dvb\_usb\_rtl28xxu rtl2823}

	There are more permanent ways of fixing the problem but I wanted to start
	with this as it is non-permanent and will default back to normal on a restart. 

	The new file in the RTL-SDR will allow you to hear FM stations. But 
	they are extremely faint. I'm going to continue to play with the file settings to
	see if I can get a clearer transmission. 
	\newpage
	\section{7-2-2015}

	I began the day with a quick interlude into learning LaTex. After about an hour
	and a half of studying I was able to produce the document you are currently reading.
	I'm not making use of many libraries but I believe it already seems more 
	organized and official than my traditional notes. Also, it is much more portable
	than a normal word document and less prone to formating failures seen in google docs. 
	
	Yesterday I got the SDR to recieve FM stations. However, they were staticy and sounded
	slowed down. I'm not sure if this is an error in my demodulation values or just that
	the computer that I'm using is too slow. I tried using a different one but that ended 
	up being slower than I remembered. I will try to get linux installed on my laptop later
	tonight when I'm done working, but I'm not confident it will work as I had trouble in the past. 

	Today I will try to replicate the results I had yesterday, but through the use of the python
	libraries instead of relying on the GNURadio GUI interface. A small note about the gui interface.
	I did learn that its possible to make variables, attach them to GUI widgets, and then alter then live. 
	It appears that having more than 3 of these can severely impact performace. To use the GUI widgets,
	simple drag one onto your workspace (I used the Wx widgets) and then give it a unique name, and 
	appropriate value range (if its a slider). Then in the blocks for different components, you can use 
	these variable names instead of hardcoded values. This is useful for gains and frequencies. 

	/paragraph{GNURadio in Python}

	I will be following the tutorial found \href{http://gnuradio.org/redmine/projects/gnuradio/wiki/TutorialsWritePythonApplications}{here}. I will begin with the dial tone generator.
	\lstset{tabsize=2, caption=dial\_tone.py,}
	\lstinputlisting[language=Python]{../RTL-SDR/dial_tone.py}

	\paragraph{Static}
	I'm still only getting static when I run the RTL-SDR. I was able to use the GNURadio toolchain from just python
	no problem, however I still need to implent the RTL-SDR from the python toolchain. I used an additional
	tool put out by osmocom but that did not product better results. Automatic gain was on, so its possible I need 
	to manually set the gain, but I am not sure. It could also be the poor quality of the antenna. the command was:
	\shellcmd{rtl\_fm -f 96.3e6 -M wbfm -s 200000 -r 48000 - | aplay -r 48k -f S16\_LE}

	aplay is a command line utility for playing audio. I had to run that separately as piping the data did not seem to work. I got 	      this command from \href{http://sdr.osmocom.org/trac/wiki/rtl-sdr}{osmocom's website}		

	\section{07-03-2015}
	
	I will not have much time to work today due to the holiday weekend. I will try to get Linux installed on 
	my laptop in my free time. Either tomorrow or next week I will try to find a python example using the RTL-SDR
	driver. After that, I may switch gears back to the BladeRF. 
	
	\section{07-04-2015}
	Happy 4th of July! No progress will be made today as I'll be busy all day. 

	\section{07-05-2015}
	Didn't get a chance to work today either. Family event. Will continue progress tomorrow. 

	\newpage
	\section{07-06-2015}
	Back to work today. I was attempting to install a program called Gqrx. Using the instructions from their github
	I added a repository and installed just the gqrx program. However, this created a massive conflict in my repositories. I 
	uninstalled and figured I would just go back to working on the FM model to see if I could be sure its working right. However,
	installing that package messed up my installation of gnu radio. So now I must uninstall and reinstall evertyhing. At the time, I was following this tutorial \href{https://github.com/mossmann/hackrf/wiki/Installing-gnuradio-on-Ubuntu-14.04-with-the-packaging-manager}{On Gqrx}. I found a person with a similar problem to mine \href{https://github.com/kpreid/shinysdr/issues/12}{here}. 
	There are instructions below on how to reinstall everything. I will be trying that next and will hopefully get back to where
	I was. It may also be time to start learning more about virtuan environments or just how to reimage a computer quickly.	 

	Currently trying to use the following command to reset the broken packages. This is taking a long time so I will have to wait
	for it to finish in order to continue reinstalling gnuradio. The command is:\shellcmd{sudo apt-get dist-upgrade}	
	This did not end up working. Next I tried using aptitude, which as I have read, will work to fix these broken dependencies and
	packages. The command for that is: \shellcmd{sudo aptitude install <packagename>}
	Still trying to get everything installed and working. The repository made GNURadio work again, but I lost the Osmocom
	packages. Still trying. I have been following the instructions direct from osmocom but I'm getting errors related to
	"Gruel" when I try to use cmake. 
	\paragraph{Success}
	Finally remembered that the instructables I was using previously had the commands listed (although for arch). The three
	necessary components can be installed by using:\shellcmd{sudo apt-get install gnuradio}\shellcmd{sudo apt-get install rtl-sdr}\shellcmd{sudo apt-get install gr-osmosdr}
	As a note, the instructable used arch linux and the final package was installed as gr-osmosdr-git. Debian/Fedora/Arch users
	may need to use this form of the package instead of gr-osmosdr. Another github was found \href{https://github.com/csete/gnuradio-grc-examples}{here} with plenty of examples. They may be out of date as the AM example did not compile correctly. Another useful site was \href{https://tapiovalli.wordpress.com/2014/08/02/rtl-sdr-gnu-radio-and-building-my-own-am-receiver/}{found here} this site had
	a single example, but it seemed to be working. I'm going to try to test to see if I can use my CB radio with this. I 
	think it will be too low of a frequency, but this would make testing the blade very easy as I can control the transmission
	with just the hand held radio. 
	\paragraph{R820T} So after trying to find a software solution for determining which tuner the device was using, I decided 
	that the smarter solution was to just open the SDR up. The chip inside is the Rafael Micro R820T. This is GOOD. Next to the
	elusive E4000 this is the best possible tuner to have. Its range is 24-1766 MHz. Not quite as high as the E4000 but much
	lower than that one and much higher than any other tuner that it could have had. I'm pretty happy. The device came apart with
	no tools or fuss, everything was just snapped together. The device even went back together when it was done!
	\paragraph{CB Radio} After a bit of struggling it looks like I'm able to pick up CB signals. The main issue I have now
	is that without better knowledge of signal processing I can't properly adjust the FIR filter. So I am able to recieve
	CB signals but they are not specific to any channel. The antenna that comes with this RTL-SDR is also awful. If I
	transmit over wireless I recieve really garbled sounds. If I unscrew both antennas and connect by contact it is crystal
	clear but still not on any specific channel. I don't think I'll spend much more time on this however. The file is on github
	under the name airband\_4.grc and airband\_5.grc. The 5 was an attempt to fix some of the issues with the downloaded file.
	It was called airband\_4 on the website because it was able to pickup 4 different AM broadcasts used by airplanes to
	communicate. 
	\\\\
	I found two resources to keep track of for general SDR and DSP information.
	\begin{itemize}
		\item http://complextoreal.com/tutorials/\#.VZr6nKH7s\_t
		\item http://greatscottgadgets.com/sdr/
	\end{itemize}
	
	\newpage
	\section{07-07-2015}
	Today I will be taking a look at the bladeRF SDR. The first problem that comes up is it appears I forgot to bring 
	a compatible antenna with me. I have to double check all my stuff to be sure, but due to the typically large size of
	these antennas I think it is likely that they got left behind. 
	Anyway, despite not having an antenna I'm going to push on anyway and then see what I have lying around. I should be able
	to order something on Amazon fairly quickly if needed anyway. Nuand is nice enough to have a fairly detailed set of
	documents on their github account. It also looks like these may have been updated since the last time I went through this
	process. The github account is found \href{https://github.com/nuand/bladeRF/wiki#Getting\_Started}{here}. I also just saw on
	this webpage that there is now matlab support for the bladeRF on windows. I'm not sure if this was always a feature 
	or if this is indeed new to the program. Eitherway it is something to be aware of. It has support for Simulink too. 
	
	Following the tutorial, the first step is to add the appriproate repository so we can download the files. There is a 
	snapshot section that allows you to get the most up to date releases but these are usually far from stable. The commands
	used are. 
	\subsection{Install BladeRF}
	The commands are:
	\shellcmd{sudo apt-get-repository ppa:bladerf/bladerf}
	\shellcmd{sudo apt-get update}
	\shellcmd{sudo apt-get install bladerf}

	Of course I immediately get an error saying "you have held broken packages." I should probably do more research into how
	to avoid this scenario, as it seems to come up quite a lot in working with GNURadio. This could be related to using the
	other PPA from before. I'm going to try to use aptitude to install this and see if that helps. Aptitude seems to be able
	to solve the problem, however it will be removing:
	\begin{itemize}
		\item gr-osmosdr (which I know I need)
		\item libbladerf0
		\item libgnuradio-osmosdr0.0
	\end{itemize} 
	So I know that all three of these libraries are needed to run the blade and all but the libbladerf0 are needed to also
	run the RTL-SDR. I'll give it a shot and see what happens, I can always reinstall from the repository (I Hope). After running 		\shellcmd{sudo aptitude install bladerf} and accepting their suggestions it seems to have installed properly. I also ran:
	\shellcmd {sudo apt-get install libbladerf-dev} but it seemed to have already been installed. Now we can install the 
	updated FPGA drivers. 
	\subsection{Download new firmware}
	The commands are:
	\shellcmd{sudo apt-get install bladerf-firmware-fx3}
	\shellcmd{sudo apt-get install bladerf-fpga-hostedx40}
	\shellcmd{sudo apt-get install bladerf-fpga-hostedx115} 
	It's worth noting that we have the x40 not the x115 so we only need to install the x40. 
	\subsection{Reinstall GNURadio}
	The next step seems to indicate that
	I should build the gnuradio from sources and not use the version found in the repository. Luckily, BladeRF links to a 
	download script that pretty much puts cruise control on. Hopefully you have luck too. Here are the commands. 
	\shellcmd{mkdir -p ~/software/gnuradio-build}
	\shellcmd{cd ~/software/gnuradio-build}
	\shellcmd{wget http://www.sbrac.org/files/build-gnuradio}
	\shellcmd{chmod +x ./build-gnuradio}
	\shellcmd{./build-gnuradio -m prereqs gitfetch}

	It took about 15-20 minutes to finish but everything downloaded successfully. Afterwards its necessary to actually build
	the downloaded programs. I followed the steps outlined to build GNURadio and the program halts halfway through with a 
	cryptic error message. Hopefully I can diagnose the problem because it took roughly an hour to get that far.  
	 	
	\paragraph{Success!} I ended up needing to use the Pybombs tool. It worked like a charm. This is linked to in the github
	wiki mentioned above and also found on the GNUiRadio webpage.

	\newpage
	\section{07-08-15}
	Today I will be traveling and not have much time to work. I spoke too soon on the success of pybombs. It looked like it
	worked and it did install the BladeRF software properly, however the GNURadio software did not. I tried several times
	and at one point got an error about the jdk which wasn't listed as a prereq. I tried the plain build again and got much
	further (89\% vs 56\% originally). I'll try running it again and see if it goes any further. 

	\section{07-09-15}
	Today's focus switched to some open tasks I had with Dr. Integlia. The first is a comparison of various FPGA platforms
	available. After looking at everything, I am most excited by the offerings from digilent, especially the Zynq based boards.
	These have an onboard FPGA and a microprocessor onboard. I also prepared a comparison of SDR platforms. One board even 
	had a Zynq SoC on board that was programmable. This could lead to interesting research down the line as students become
	better at using the FPGAs. The FPGA comparison is available in this repository but for now the SDR is not. I will try
	to migrate it here soon too.

	\section{07-10-15}
	Today I started the Literature review. There is a lot of ground to cover. I got through 14 sources. So far the general
	gist of what I've read is that the USRP is the go to SDR for research, but that people are excited by the RTL-SDR. 
	GNU Radio seems to be very well recieved for both simulation and real world testing. I'm curious to see if some of the
	other software tools identified have a following in the academic world too. 

	I also learned how to use the IEEEtrans format for Latex and how to use Bibtex.

	\section{07-11-15}
	I won't have as much time to work today but I will try to get some amount done. I need to port the FPGA comparison
	to excel and continue the Lit Review. I will be trying to spend most of the afternoon working through Node.js and 
	research graduate schools. I've been keeping up on Node.js fairly well but have been falling behind on Grad school 
	searches. 

	\section{07-12-15}
	Today I will be converting the Latex notes into an excel spreadsheet for sharing. I may not have much time to work
	today or tomorrow but I will put full days on Tuesday and Wednesday. 
\end{document}


