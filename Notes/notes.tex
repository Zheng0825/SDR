\documentclass{article}

\newcommand{\shellcmd}[1]{\indent\indent\texttt{\footnotesize\# #1}\\}
\usepackage{hyperref}
\usepackage{listings}
\usepackage[margin=0.5in]{geometry}

\title{Notes on SDR}
\date{2015-07-01}
\author{J.D. McCormack}

\begin{document}
	\pagenumbering{gobble}
	\maketitle
	\newpage
	\pagenumbering{arabic}
	
	\section{Introduction}

	The following document will contain information related to my efforts in software defined radio.
	From time to time, the document may include information on other topics as well. It will initially
	serve as a day to day journal of my activites. After a certain discovery phase, this document will
	be updated to create a manual or tutorial for future students to learn from. It is not recommended
	that the document be used for self study until a newer version is created. Many of the steps 
	that are outlined will be dead ends or poor practices while the disocvery process takes place. 

	\newpage

	\section{7-1-2015}

	Currently working on using the RTL SDR. 

	What I have learned so far: 
		\begin{itemize}
			\item This SDR is different than a FUNCUBE dongle
			\item It will work from around 500 Hz to 1700 Hz
			\item It can only RECIEVE 
		\end{itemize}
	The fact that this can only recieve is by far the largest setback.
	This will ultimately not be what we can use for cognitive radio. 

	However, the tutorials still seem more widely available than those
	found for the BladeRF which is bi-directional. Therefore I will
	continue for a bit. It may serve a purpose as part of the larger 
	sensor network at some point. We could still use it to relay information
	over a large distance. 

	There are "Cubelites" being sent out that are basically funcube dongle
	based satellites. These sound pretty cool. 

	I want to start taking notes in Latex so they will be more readable than
	a plain .txt file and still useable with version control. 

	I need to install texlive-full and texmaker. 

	Currently I am set back by needing to update ubuntu. I can't install
	anything else while the updates are installing. 
	
	\paragraph{Success}
		
		
	So far I have made decent progress. The RTL is working with GNU SDR. I had trouble
	at first when the kernel was loading a separate driver that bogs it down. 

	I used this command to fix that problem:

	\shellcmd{sudo rmmod dvb\_usb\_rtl28xxu rtl2823}

	There are more permanent ways of fixing the problem but I wanted to start
	with this as it is non-permanent and will default back to normal on a restart. 

	The new file in the RTL-SDR will allow you to hear FM stations. But 
	they are extremely faint. I'm going to continue to play with the file settings to
	see if I can get a clearer transmission. 
	\newpage
	\section{7-2-2015}

	I began the day with a quick interlude into learning LaTex. After about an hour
	and a half of studying I was able to produce the document you are currently reading.
	I'm not making use of many libraries but I believe it already seems more 
	organized and official than my traditional notes. Also, it is much more portable
	than a normal word document and less prone to formating failures seen in google docs. 
	
	Yesterday I got the SDR to recieve FM stations. However, they were staticy and sounded
	slowed down. I'm not sure if this is an error in my demodulation values or just that
	the computer that I'm using is too slow. I tried using a different one but that ended 
	up being slower than I remembered. I will try to get linux installed on my laptop later
	tonight when I'm done working, but I'm not confident it will work as I had trouble in the past. 

	Today I will try to replicate the results I had yesterday, but through the use of the python
	libraries instead of relying on the GNURadio GUI interface. A small note about the gui interface.
	I did learn that its possible to make variables, attach them to GUI widgets, and then alter then live. 
	It appears that having more than 3 of these can severely impact performace. To use the GUI widgets,
	simple drag one onto your workspace (I used the Wx widgets) and then give it a unique name, and 
	appropriate value range (if its a slider). Then in the blocks for different components, you can use 
	these variable names instead of hardcoded values. This is useful for gains and frequencies. 

	/paragraph{GNURadio in Python}

	I will be following the tutorial found \href{http://gnuradio.org/redmine/projects/gnuradio/wiki/TutorialsWritePythonApplications}{here}. I will begin with the dial tone generator.
	\lstset{tabsize=2, caption=dial\_tone.py,}
	\lstinputlisting[language=Python]{../RTL-SDR/dial_tone.py}

	\paragraph{Static}
	I'm still only getting static when I run the RTL-SDR. I was able to use the GNURadio toolchain from just python
	no problem, however I still need to implent the RTL-SDR from the python toolchain. I used an additional
	tool put out by osmocom but that did not product better results. Automatic gain was on, so its possible I need 
	to manually set the gain, but I am not sure. It could also be the poor quality of the antenna. the command was:
	\shellcmd{rtl\_fm -f 96.3e6 -M wbfm -s 200000 -r 48000 - | aplay -r 48k -f S16\_LE}

	aplay is a command line utility for playing audio. I had to run that separately as piping the data did not seem to work. I got 	      this command from \href{http://sdr.osmocom.org/trac/wiki/rtl-sdr}{osmocom's website}		

	\newpage
	\section{07-03-2015}
	
	I will not have much time to work today due to the holiday weekend. I will try to get Linux installed on 
	my laptop in my free time. Either tomorrow or next week I will try to find a python example using the RTL-SDR
	driver. After that, I may switch gears back to the BladeRF. 
	
	\newpage
	\section{07-04-2015}
	Happy 4th of July! No progress will be made today as I'll be busy all day. 
\end{document}

