\documentclass{article}

\title{Notes on SDR}
\date{2015-07-01}
\author{J.D. McCormack}

\begin{document}
	\pagenumbering{gobble}
	\maketitle
	\newpage
	\pagenumbering{arabic}
	
	\section{Introduction}

	The following document will contain information related to my efforts in software defined radio.
	From time to time, the document may include information on other topics as well. It will initially
	serve as a day to day journal of my activites. After a certain discovery phase, this document will
	be updated to create a manual or tutorial for future students to learn from. It is not recommended
	that the document be used for self study until a newer version is created. Many of the steps 
	that are outlined will be dead ends or poor practices while the disocvery process takes place. 

	\newpage

	\section{7-1-2015}

	Currently working on using the RTL SDR. 

	What I have learned so far: 
		\begin{itemize}
			\item This SDR is different than a FUNCUBE dongle
			\item It will work from around 500 Hz to 1700 Hz
			\item It can only RECIEVE 
		\end{itemize}
	The fact that this can only recieve is by far the largest setback.
	This will ultimately not be what we can use for cognitive radio. 

	However, the tutorials still seem more widely available than those
	found for the BladeRF which is bi-directional. Therefore I will
	continue for a bit. It may serve a purpose as part of the larger 
	sensor network at some point. We could still use it to relay information
	over a large distance. 

	There are "Cubelites" being sent out that are basically funcube dongle
	based satellites. These sound pretty cool. 

	I want to start taking notes in Latex so they will be more readable than
	a plain .txt file and still useable with version control. 

	I need to install texlive-full and texmaker. 

	Currently I am set back by needing to update ubuntu. I can't install
	anything else while the updates are installing. 
	
\end{document}
