\section{RTL-SDR}

	The RTL-SDR is a recent tool discovered by the DIY and hacking community. Its original purpose is to be used as a digital
	TV tuner. However, it was discovered that this system could also be used to general SDR purposes. There is now a large 
	community dedicated to using this tiny SDR to recieve various different signals. Prior to the creation of the open source
	drivers for the RTL-SDR, the most popular devices for SDR came from USRP. The USRP devices are fantastic products, but cost
	at least \$1,000 and can cost quite a bit more with additional features. The RTL-SDR is based on rhe Realtek RTL2832U chip.
	This device can often be purchased for between \$20 and \$30 \cite{6526525}. The range for the RTL-SDR is typically 64 MHz
	to 1700 MHz, however this varies depending on which tuner the manufacturer paired with it. The authors in \cite{6526525} 
	paired the RTL-SDR with a mixer in order to lower the range all the way to DC. For this, they used the NE6062AN chip.  
	Starting with Release 2013b, MATLAB/Simulink now have a support package that targets RTL-SDR devices. In Simulink, the
	package contains a single block called "RTL-SDR Receiver." This block allows the user to tune the center frequency, 
	change the tuner gain, set the sampling rate, and alter the frequency correction factor. The block can then output
	the complex envelope (IQ) of the recieved signal in both floating point and integer formats\cite{6893337}. 
	
	 
