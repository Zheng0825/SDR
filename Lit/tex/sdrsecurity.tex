\section{Security in SDR and CR Networks}

A major portion of our work revolves around what we believe is a new concept in SDR security. SDR and 
cognitive radio devices suffer from traditional security flaws of wireless transmission in addition to
many of their own. The authors in \cite{1400381} present their idea of securing an SDR transmission
using a Gigabit Ethernet AES (Advanced Encryption Standard) Encryption Engine.  This is implemented on 
an FPGA. AES is a standard means of encryption that uses cipher keys to hide data. In \cite{1285401} the 
author's discuss concerns in validating whether an SDR system is secure or not. They discuss separating the 
certification into two separate blocks so the hardware and software can be evaluated separately. Peter Hillmann and Bjorn Stelte have presented a unique way to share cryptographic keys over SDR systems \cite{6694489}. Their system uses the Secure Communications Interoperability Protocols (SCIP). In NATO's current
configuration, each user has a public and private cryptographic key component. In the proposed system, each SDR has a minimum of two available channels, "x" and "y", where one is used for local operations such as 
voice and data and the other is used to relay communications data to and from other units. A secure 
communication channel is established by borrowing time and bandwidth on channel "y" for a certificate
based asymmetric key negotiation with a lead radio. Once the secure link is established, the symmetric net
key is transmitted to the new users. This will then allow for the new radios to communicate fully on "x" and
"y". The paper continues to discuss numerous methods employed for ensuring the secure transfer 
of keys. 
